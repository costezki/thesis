\chapter{The grammar structure}
\label{ch:the-grammar}

Now that the main theoretical foundations have been covered, I turn to describe the grammatical units  system networks adopted in this thesis some being from Sydney and others from Cardiff grammars. They have common parts and also differ in large parts on their paradigmatic and syntagmatic descriptions. Just like in the previous chapter I base my discussion on pragmatic grounds.

First I conduct a critical discussion on the differences between the important unit structures in Sydney and Cardiff grammars: the clause, the verbal group, nominal group, the adjectival and adverbial groups.
And then I describe the two important system networks: TRANSITIVITY  and MOOD. The formed is from Cardiff grammar and the latter belongs to Sydney grammar. 


\section{The grammatical units for parsing}
\label{sec:discussion-unit-classes}

I turn now to discuss the structure of the units implemented in the Parsimonious Vole parser. I provide arguments and reasoning for choosing one over the other unit structure. The general principle for selection is that some unit structures are closer to the traditional syntactic analysis and thus easier to parse and the other ones may be a level of abstraction higher falling on the semantic grounds and thus becoming more difficult to capture in structural variance and requiring additional lexico-semantic resources.

%For the reasons of limited space I skipped introducing the Sydney and Cardiff grammars and in turn assume that the reader is familiar with the details fo both of them. And for a general overview of the unit structure in each of the grammars please refer to Appendix \ref{ch:syntax-overview}. 
%Nevertheless, as it is a parallel contrastive discussion, even if the reader is familiar with one grammar only I hope it becomes clear how does certain phenomena are dealt with in the other one. 

\subsection{Verbal group and clause boundaries}
\label{sec:verbal-grpoup-and-clause-division}
In Sydney Grammar the verbal group is described as an expansion of a verb just like the nominal group is the expansion of the noun\citep[396]{Halliday2013}. There are certainly words that are closely related and syntactically dependent on the verb all together forming a unit that functions as a whole. For example the auxiliary verbs, adverbs or the negation particles are words that are directly linked to a lexical verb. The verb group functions as Finite + Predicator elements of the clause in Mood structure and as Process in Transitivity structure. 

In Cardiff Grammar the verb group is dissolved moving the Main Verb as the pivotal element of the Clause unit. All the elements that form the clause structure and those that form the verb group structure are brought up together to the same level as elements of a clause. The clause structure in Cardiff Grammar comprises elements with clause related functions(like Subject, Adjunct, Complement etc.) and other elements with Main Verb related functions(Auxiliary, Negation particle, Finite operator etc.).

Regarded from the Hallidayan rank scale perspective, merging the elements of the verb group into clause structure is not permitted because the units are at different ranks. However it is not a problem for the relaxed rank scale version presented in subsection \ref{sec:rank-system}. The reason for adopting such an approach is best illustrated via complex verb groups with more than one non-auxiliary verb such as in examples \ref{ex:complex-verb-groups1}--\ref{ex:complex-verb-groups3}. 

I begin by addressing the impact of this merger on (a) the clause structure (b) the clause boundaries and (c) semantic role distribution within the clause.

\begin{exe}
	\ex\label{ex:complex-verb-groups1}
	(The commission \textbf{started to investigate} two cases of overfishing in Norway.)  
	\ex\label{ex:complex-verb-groups2}
	(The commission \textbf{started} (\textbf{to investigate} two cases of overfishing in Norway.))
	\ex\label{ex:complex-verb-groups3}
	(The commission \textbf{started} (\textbf{to finish} (\textbf{investigating} two cases of overfishing in Norway.)))
\end{exe}

%Then answers to the above questions boil down to whether we allow for more than one lexical verb per predicate or not. 
In Sydney Grammar ``started to investigate'' (in \ref{ex:complex-verb-groups1}) is considered a single predicate of investigation which has specified the aspect of event incipiency despite the fact that there are two lexical verbs within the same verbal group. The ``starting'' doesn't constitute any kind of process in semantic terms but rather specifies aspectual information about the investigation process. 
This is argued by looking at the conditions on the participants and it is equivalent in a formal approach to looking at where the selection restrictions for complements come from.
The boundaries of the clause governed by this predicate stretch to the entire sentence.

Semantically it is a sound approach because despite the presence of two lexical verbs there is only one event. However allowing such compositions leads to unwanted syntactic analysis for multiple lexical verb cases in examples such as \ref{ex:complex-verb-groups3}. To solve this kind of problem Fawcett dismisses the verb groups and merges their elements into clause structure. He proposes the syntactically elegant principle of \textit{one main verb per clause} \citep{Fawcett2008}. Applying this principle to the same sentence yields a structure of two clauses illustrated in example \ref{ex:complex-verb-groups2} where the main clause is governed by the verb ``to start'' and the embedded one by the verb ``to investigate''. Note the conflict between ``one main verb per clause'' with Halliday's principle that only whole units form the constituency of others (the (c) principle of rank scale described in subsection \ref{sec:rank-system}). So allowing incomplete groups into the constituency structure would breach the entire idea of unit based constituency. 

Semantically the clause in SFL is a description of an event or situation as a figure with a process, participants and eventually circumstances where the process is realised through a lexical verb. Looking back to our examples, does the verb ``to start'' really describes a process or merely an aspect of it? Halliday treats such verbs as aspectual and when co-occurring with other lexical verbs they are considered to form a single predicate. Accommodating Fawcett's stance, mentioned above and contradicting Halliday's approach, requires weakening the semantic requirement and allowing aspectual verbs to form clauses that contribute \textit{aspectually or modally} to the embedded ones. I mention also the modal contribution because some verbs like \textit{want, wish, hope} and others behave syntactically like the aspectual ones. Moreover, Fawcett introduces into the Cardiff Grammar Transitivity network an \textit{influential} process type including all categories of meanings that semantically function as process modifiers: tentative, failing, starting, ending etc.

I adopt here Fawcett's ``one main verb per clause'' principle which as a consequence changes the way clauses are partitioned, leads to abolition of the verbal group and introduces the ``influential'' process type. Next I discuss it's impact on the structure of the clause unit. 

\subsection{The Clause}
\label{sec:cardiff-clause}
It is commonly agreed in linguistic communities that the unit of the clause is one of the core elements in human language. 
%It is considered the syntactic unit that expresses semantic units of a situation referring to a potentially rich array of meanings. The clause structure has been studied for long time and 
The main clause constituents are roughly the same in SFL as the ones in the traditional grammar \citep{Quirk1985}, transformational grammar \citep{Chomsky1957} an indirectly in dependency grammar \citep{Hudson2010}.

In current work I adopt the Cardiff Grammar clause structure with the \textit{Main Verb} as pivotal element. Though there is no element that is obligatorily realised in English, I consider in the current work, the Main Verb to be way to flag a clause. In SFL are described clauses without a main verb such as minor clauses (exclamations, calls, greetings and alarms) that occur in conversational contexts and elliptical clauses \citet{Halliday2013} such as the one in example \ref{ex:elipted-clause}, none of which are covered in the present work.

\begin{exe}
	\ex\label{ex:elipted-clause} They were in the bar, \textit{Dave in the restroom and Sarah by the bar}.
\end{exe}

%As mentioned before in Definition \ref{def:structure} the elements of a structure are defined in terms of their function contributing to the formation of the whole unit.
I adopt Cardiff Grammar clause structure for English. It is formed of the \textit{Subject}, \textit{Finite}, \textit{Main Verb}, up to two \textit{Complements} and a various number of \textit{Adjuncts}. All the elements of the assumed verbal group are part of the clause as well such as Auxiliary Verbs, Main Verb Extensions, Negators etc. (see Appendix \ref{ch:syntax-overview} for a 
complete list).

%%TODO insert the complete list as done for nominal group below

%\todo*{}{TODO continue, maybe mention about the clause complexing}
%SFG accounts for how clauses form nexuses of tactic relations (see Definition \ref{def:taxis}).


\subsection{The Nominal Group}
\label{sec:nominal-group}

The nominal group expresses things, classes of things or a selection of instances in that class. This section argues for adoption of the Sydney grammar noun group structure with a specific extension. It is that the elements of the nominal group can be filled, in addition to word classes, by the groups as well. This possibility is opened by the rank scale relaxation (Section \ref{sec:rank-system}) and Cardiff embedding principle (Definition \ref{def:embedding}). In addition to that I argue below for working on semantic and syntactic heads in two steps. First create the structure with the syntactic one (the Head) and then derive the semantic one (the Thing).

\begin{table}[!ht]
    \centering
	\begin{tabular}{|c|c|c|c|c|c|}
		\hline
		\textit{those} & \textit{two} & \textit{old} & \textit{electric} & \textit{trains} & \textit{from Luxembourg} \\ \hline
		\multicolumn{4}{|c|}{Pre-Modifier}                               & Head            & Post-Modifier            \\ \hline
		Deictic        & Numerative   & Epithet      & Classifier        & Thing           & Qualifyier               \\ \hline
		determiner     & numeral      & adjective    & adjective         & noun            & prepositional phrase     \\ \hline
	\end{tabular}
	\caption{An example of a nominal group in the Sydney Grammar \citep[264]{Halliday2013}}
	\label{tab:example-ng}
\end{table}

In  Table \ref{tab:example-ng} an example analysis is presented of the nominal group proposed in the Sydney grammar \citep[364--369]{Halliday2013}. The Sydney nominal group is constituted by a head nominal item modified by descriptors or selectors such as: \textit{Deictic}, \textit{Numerative}, \textit{Epithet}, \textit{Classifier}, \textit{Thing} and \textit{Qualifier}. Each element has a fairly stable correspondence to the word classes, expected to be expounded by lexical items. Table \ref{tab:function-pos-mapping} presents the mappings between the elements of nominal group and the word classes. 

\begin{table}[h]
        \centering
	\begin{tabulary}{0.9\linewidth}{|C|C|}
		\hline
		\textbf{Experiential function in noun group} & \textbf{class (of word or unit)} \\ \hline
		Deictic                             & determiner, predeterminer, pronoun, adjective \\ \hline
		Numerative                          & numeral(ordinal or cardinal) \\ \hline
		Epithet                             & adjective \\ \hline
		Classifier                          & adjective, noun \\ \hline
		Thing                               & noun                         \\ \hline
		Qualifier                           & prepositional phrase, clause \\ \hline
	\end{tabulary}
	\caption{Mapping of noun group elements to classes \citep[379]{Halliday2013}}
	\label{tab:function-pos-mapping}
\end{table}

Inspired from Cardiff grammar, in addition to word classes, the elements of the nominal group can also be filled by the group classes corresponding to each word class above. This way the Numerative, in addition to words, can be filled by a noun group, Epithet by an adjectival group, Classifier by an adjective or noun group and finally each of the elements can be filled by a coordination group discussed in Section \ref{sec:coordination}.

%There is little variation in how these functions are linked to the word classes. However the variation is not provided in the Sydney Grammar as it is in Cardiff Grammar. The latter provides data driven filling probabilities for each functional element to a set of possible unit classes \citep{Fawcett2000}.

The elements in Cardiff Grammar differ from those of Sydney Grammar. Table \ref{tab:carfiff-ng} exemplifies a noun group analysed with Cardiff Grammar covering all the possible elements. Table \ref{tab:cg-mappings} provides a legend for the Cardiff Grammar acronyms along with mappings to unit and word classes that can fill each element.

\begin{table}[H]
	\resizebox{\linewidth}{!}{
		\begin{tabular}{|c|c|c|c|c|c|c|c|c|c|c|c|c|c|c|}
			\hline
			\textit{or} & \textit{a photo} & \textit{of} & \textit{part} & \textit{of} & \textit{one} & \textit{of} & \textit{the best} & \textit{of} & \textit{the} & \textit{fine} & \textit{new} & \textit{taxis} & \textit{in Kew} & \textit{,} \\ \hline
			\multicolumn{12}{|c|}{pre-modifiers} & head & \multicolumn{2}{c|}{post-modifiers} \\ \hline
			\& & rd & v & pd & v & qd & v & sd or od & v & dd & m & m & h & q & e \\ \hline
		\end{tabular}}
		\caption{The example of a nominal group in Cardiff Grammar}
		\label{tab:carfiff-ng}
	\end{table}
	%
	\begin{table}[h]
		\begin{tabulary}{\textwidth}{|C|C|C|}
			\hline
			\textbf{symbol} & \textbf{function meaning} & \textbf{class (of word or unit)} \\ \hline
			rd & representational determiner & noun, noun group \\ \hline
			v & selector ``of'' & preposition \\ \hline
			pd & partitive determiner & noun, noun group \\ \hline
			fd & fractional determiner & noun, noun group, quantity group \\ \hline
			qd & quantifying determiner & noun, noun group, quantity group \\ \hline
			sd & superlative determiner & noun, noun group, quality group, quantity group \\ \hline
			od & ordinative determiner & noun, noun group, quality group \\ \hline
			td & tipic determiner & noun, noun group \\ \hline
			dd & deictic determiner & determiner, pronoun, genitive cluster \\ \hline
			m & modifier & adjective, noun, quality group, genitive cluster \\ \hline
			h & head & noun, genitive cluster \\ \hline
			q & qualifier & prepositional phrase, clause \\ \hline
			\& & linker & conjunction \\ \hline
			e & ender & punctuation mark \\ \hline
		\end{tabulary}
		\caption{The mapping of noun group elements to classes in Cardiff grammar}
		\label{tab:cg-mappings}
	\end{table}
	%
	
The elements in Cardiff Grammar are based on semantic criteria supported by lexical and syntactic choices. Consequently some elements cannot be derived based on solely syntactic criteria, requiring semantically motivated lexical resources. Semantically bound elements which are a challenge are predominantly determiners \textit{Representational, Partitive, Fractional, Superlative, Typic Determiners} while the rest of the elements: \textit{Head, Qualifier, Selector, Modifier and Deictic, Ordinative and Quantifying Determiners} can be determined solely on the syntactic criteria. The latter correspond fairly well to the Sydney version of nominal group which is adopted in the present work with the benefits of the relaxed rank system replacing the sub-structures with embedded units and simplifying the syntactic structures. 
	
	%3. introduction of negation element, linker, punctuation
	%``No'' as determiner. negative pronoun noone.
	%\citep[62,109,185]{Quirk1985}, \citep[365--374]{Halliday2013}, 
	%
	%The negation element is in Cardiff Grammar in clause structure so that it is separated from the finite. It's adoption in noun structure might or might not be a good idea. It is useful for complex groups. 
	%
	%\begin{exe}
	%\ex \label{ex:example-of-no}
	%No breathing man or animal can escape that forest alive.
	%\end{exe} 
	
Another simplification is renouncing to distinction between the Head and Thing \citep[390--396]{Halliday2013} discussed in Section \ref{sec:heads}. Thus if the logical Head of the nominal group is a noun then it is labelled as the Thing leaving the semantic discernment as a secondary process and out of the current scope. Otherwise, in cases of nominal groups without the Thing element, if the Head is a pronoun (other than personal), numeral or adjective (mainly superlatives) then they function as Deictic, Numerative or Epithet. So, as will be described in Chapter \ref{ch:parsing-algorithm}, I propose to parse the nominal groups in two steps: first determine the main constituting chunks and assign functions to the unambiguous ones and second perform a semantically driven evaluation for the less certain units. 
%However, in the present work the second step has not been covered. 
	
%	\todo*{probably remove: discussion of Head/Thing and Cardiff Grammar determiners distinction}{
Next I explain the two step process using for illustration cases when the Thing is present but it is different from the Head such as in examples \ref{ex:dectic-ngs}--\ref{ex:dectic-ngs2}. 
\begin{exe}
    \ex \label{ex:dectic-ngs} (a cup) of (tea)
    \ex \label{ex:dectic-ngs1}(some) of (those youngsters)
    \ex \label{ex:dectic-ngs2}(another one) of (those periodic eruptions)
\end{exe}

These nominal groups can be analysed in two ways. Either  being about the ``cup'', ``some'' or ``another one'' leading to a structure where the first noun is the head succeeded by a prepositional phrase Qualifier; or rather about ``tea'', ``youngsters'' and ``eruptions'' where the second noun is the head and so adopting a structure with complex determiners.

Table \ref{tab:exmaple-analisys-parsing-syn-sem-heads} shows on the first row an analysis with syntactic head i.e. the Head defined in Sydney grammar and on the second row an analysis with semantic head i.e. the Head defined in Cardiff grammar that also coincides with the Thing from Sydney grammar.

The syntactic Head is always the first noun in the nominal group. 
In the semantic evaluation phase special attention is given to Qualifiers filled by prepositional phrases starting with ``of'' preposition and whether the nominal group may function as qualifying, quantifying, ordination or other type of determiner. 

Cardiff Grammar weakens the assumption that every prepositional phrase acts as Qualifier in a nominal group and it the special case of the preposition ``of''. It is allowed to act not as the element introducing a prepositional phrase but as a end mark of a determiner-like selector. Thus making the former noun group a determiner in the latter one. 

If the above conditions are satisfied, in the semantic evaluation phase, then the prepositional phrase Qualifyier is disassembled, the the preposition ``of'' is ascribed as a Selector element of the nominal group (in a way an upwards transfer) and the former nominal group (syntactically headed) becomes one of the determiners. This approach shifts the noun group head into the position of semantically based Thing and erases the discrepancy problem between them. 

\begin{table}[!ht]
    \centering
    \begin{tabular}{|c|c|c|c|}
        \hline
        \textit{a} & \textit{cup} & \textit{of} & \textit{tea} \\ \hline
        Determiner & Head & \multicolumn{2}{c|}{Qualifier} \\ \hline
        \multicolumn{2}{|c|}{Quantifying Determiner} & Selector & Head/Thing \\ \hline
    \end{tabular}
    \caption{Example analysis with syntactic and semantic heads}
    \label{tab:exmaple-analisys-parsing-syn-sem-heads}
\end{table}

\begin{exe}
    \ex \label{ex:of-qualifiers}He was the \textbf{confidant} of the prime minister.
    \ex It was the \textbf{clash} of two cultures.
\end{exe}

The above explanation is not a straight forward solution. The distinction between cases when the proposition ``of'' introduces a Qualifier or ends a Selector/Deictic requires lexical-semantic informed decision answering the question ``what is the Thing that this nominal group is about?''. And there is a lot of space for variations the syntactic structure. For example in \ref{ex:of-qualifiers} (Head/Thing marked in bold) the preposition ``of'' introduces Qualifiers.
 
In this section was discussed in detail the problem of semantic and syntactic heads (started in Section \ref{sec:heads}) applied to nominal groups in particular and how to approach parsing them. I conclude that it is fairly unambiguous and straight forward to determine the structure of nominal groups according to Sydney grammar yielding a syntactically founded structure. Once such a structure is ready it serves a a proper basis for a semantic evaluation that can be performed in terms of Cardiff grammar resulting in potential restructuring of the nominal group. The current implementation of the parser only the generation of syntactic structure is implemented leaving the semantically motivated noun groups for the future works. 

%While it is easy to just assume that the first noun in the nominal group is the head. 
%
%Therefore, I propose to parse the nominal groups in two steps: first determine the main constituting chunks and assign functions to the unambiguous ones and then in the second step to perform a semantically driven evaluation for the less certain units. 
%This evaluation can be performed by further capturing the structure of nominal groups that act as Dyslectics through their lexico-syntactic realisation patterns.
%	}
	
\subsection{The Adjectival and Adverbial Groups}
	\label{sec:advectival-adverbial-groups}
%	\todo{Revise the whole subsection}

    This section introduces how the adverbial and adjectival groups are handled by the Sydney grammar and then how their equivalent quality group is structured in the Cardiff grammar. As the structure of the quality group is semantically motivated some elements may be identified still at the syntactic level whereas some other ones require a more sophisticated lexical-semantic resources. In the last part of the section I estimate the complexity of parsing some the quality group elements elements. 

	Following the rationale of head-modifier similar to the case of nominal groups, the adjectives and adverbs function as pivotal elements to form groups. The structure of adverbial and adjectival constructions is briefly covered in the Sydney grammar in terms of head-modifier logical structures without an elaborated experiential structure like in the case of nominal groups. While the adverbial group is recognised as a distinct syntactic unit, the adjectival group is treated as a special case of nominal group. %specifically as a sub-structure of Epithet or Classifier elements.
	
	\begin{exe}
		\ex\label{ex:lucky} You're \textit{a very lucky boy}.
        \ex\label{ex:lucky1} You're \textit{very lucky}.
        \ex\label{ex:lucky2} \textit{The very lucky (one)} is you.
%        \ex\label{ex:the-veryu-old} The extremely old shall pass first.
	\end{exe}
	
    
%    Recall the example analysis of \textit{some very small wooden ones} in Table \ref{tab:example-substructure-analisys-logical} and \ref{tab:example-substructure-analisys} from Section \ref{sec:rank-system}. There the Epithet \textit{very small} has a logical sub structure.

    In the environments where nominal group functions as Attribute, typically in the attributive clauses such as \ref{ex:lucky}, it can take also more contracted forms without the Thing and Deictic where the Head moves left onto the Epithet such as in example \ref{ex:lucky1}. One particularity of these nominal groups which here are distinguished as \textit{adjectival group} units is that they cannot function as subject. For the example \ref{ex:lucky2} to be grammatical, where the Attribute is in the Subject position, I had to add a determiner and eventually an unspecified nominal Head. 

    
%    For example the nominal group \textit{a very lucky boy} with non-specific Deictic and 
    
%	For example ``very lucky'' in \ref{ex:lucky1} is analysed as a short form of the nominal group ``a very lucky boy'' in \ref{ex:lucky}. Here the Epithet is the Head of the group while the non-specific Determiner together with the Thing are missing. In example ``very'' is not nominal modifier, it does not modify the missing nominal head but the adjective ``lucky'' so they constitute a head-modifier structure filling the Epithet element and as the rank scale system does not allow groups to fill elements of groups then it is described as a substructure of the nominal group.
	
	The \textit{adverbial group}, in Sydney Grammar, has an adverb as Head which may or may not be accompanied by modifying elements \citep[419]{Halliday2013}. The adverbial groups may fill modal and circumstantial adjunct elements in a clause corresponding to eight semantic classes of: time, place, four types of manner and two types of assessment. The adverbial pre-modifiers express polarity, comparison and intensification along with only one comparison post-modifier \citep[420--421]{Halliday2013}. The adjectival and adverbial group are covered by the \textit{quality group} unit in Cardiff grammar.
    
	A thorough systemic functional examination in terms of lexis has been provided for the first time by Tucker \citet{Tucker1997,Tucker1998} materialised into a lexical-grammatical systematization of adjectives and the fine grained structure of quality group. He avoids calling the group according to the word class (adjective or adverb) but rather refers to the semantic meaning of what both groups express, i.e. the quality of things, situations or qualities themselves. The qualities of things have adjectives as their head while the qualities of situations an adverb.       
	
	In Cardiff Grammar, the head of the quality group is called \textit{Apex} while the set of modifying elements: \textit{Quality Group Deictic, Quality Group Quantifier, Emphasizing Temperer, Degree Temperer, Adjunctival Temperer, Scope} and \textit{Finisher}. The quality group most frequently fills complements and adjuncts in clauses and fill modifiers and superlative determiners in nominal groups but there are also other cases found in the data. 
	
	Just like in the case of nominal group the adverbial and adjectival groups in Cardiff grammar are semantically motivated. To automatically identify elements of the quality group would according to this scheme therefore require lexico-semantic resources.
    
    I turn now to discuss some relevant affinities concerning the adverbial groups. Some adverbs are different from others at least because not all of them can be heads of the adverbial group. Usually the adverbs that cannot act as heads, such as for example \textit{very, much, less, pretty}, function as Emphasizing and Degree Temperers. The same ones also act as adjectival modifiers. A naive attempt to identify these Temperers would be to use a list of frequent words found in these functions.
    
    Other elements of the quality group like the \textit{Scoper} or \textit{Finisher} are more difficult to identify and localize as part of the group only by syntactic cues and/or lists of words because of their inherent semantic nature. The problem is similar to detecting whether a prepositional phrase is filling a qualifier element in the preceding nominal group or is filling a complement or adjunct in the clause. Not surprisingly the Scopers and Finishers are most of the time prepositional phrases. 
	
	Another issue is continuity. The question is whether a grammar shall allow at least at a syntactic level discontinuous constituents or not. And then if so how to detect all the parts of the group even if they do not stand in proximity to each other. For example, comparatives, a complex case of a quality group, could be realised in a continuous or discontinuous forms. Compare the analyses presented in Table \ref{tab:csgq1} and \ref{tab:csgq2}. In the first case the comparative structure is a continuous quality group. In the second case the comparative is dissociated and analysed as separate adjuncts. 
	
	On one hand it is not a problem treating them as two adjuncts, because that is what they are from the syntactic point of view. However, semantically as Fawcett proposes, there is only one quality group with a discontinuous realisation whose Scope element is placed in a thematic position before the Subject. 
    
	%
	\begin{table}[H]
		\centering
		\begin{tabular}{|c|c|c|c|l|c|c|}
			\hline
			\textit{I} & \textit{am} & \textit{much} & \textit{smarter} & \textit{today} & \textit{than} & \textit{yesterday} \\ \hline
			Subject & Main Verb & \multicolumn{5}{c|}{Adjunct} \\ \hline
			pronoun & verb & \multicolumn{5}{c|}{quality group} \\ \hline
			&  & Emphasizing Temperer & Apex & Scope & \multicolumn{2}{c|}{Finisher} \\ \hline
		\end{tabular}
		\caption{Comparative structure as one quality group adjunct}
		\label{tab:csgq1}
	\end{table}
	%
	\begin{table}[H]
		\centering
		\begin{tabular}{|c|c|c|c|c|c|c|}
			\hline
			\textit{Today} & \textit{I} & \textit{am} & \textit{much} & \textit{smarter} & \textit{than} & \textit{yesterday} \\ \hline
			Adjunct & Subject & Main Verb & \multicolumn{4}{c|}{Adjunct} \\ \hline
			adverb & pronoun & verb & \multicolumn{4}{c|}{quality group} \\ \hline
			&  &  & Emphasizing Temperer & Apex & \multicolumn{2}{c|}{Finisher} \\ \hline
		\end{tabular}
		\caption{Comparative structure split among two adjuncts}
		\label{tab:csgq2}
	\end{table}
	%
    
	For an automatic process to identify a complex quality group is a difficult task. It needs to pick up cues like a comparative form of the adjective followed by the preposition ``than'' and then look for two terms being compared. Given some initial syntactic structure such patterns could be modelled and applied but only as a secondary semantically oriented process.
	
	Since both the adverbial and adjectival groups have similar structures, it is syntactically feasible to automatically analyse them in terms of head-modifier structures in a first phase followed by a complementary process which assigns functional roles to the quality group components.

\section{Selected system networks for parsing}
    The previous section described the units of structure adopted in the current work. I turn now to describe a few systemic networks relevant to selected units that are fairly low in delicacy and are quite close to what is considered grammatical features in the traditional grammar. 

    %The system networks organise the grammatical features realised in structure
    
    
    
    It is close to terms and concepts of syntactic structure in traditional grammar (role and definition of Subject, Complement and other functional elements) and the . 
    
\subsection{MOOD}
\label{sec:mood}

    One of the first system networks presented in the Introduction to Functional Grammar \citep{Halliday2013} is that of MOOD. 
    This system is introduced in the discussion of interpersonal metafunction of language. In this view a clause is conceptualised as message exchanged between dialogue interactants. A range of grammatical features from traditional grammar such as \textit{mood}, \textit{modality}, \textit{aspect}, \textit{mode}, \textit{polarity}, \textit{tense} etc. are covered, under different names, in this system network. 
    
    The terms in SFL literature sometimes are capitalised to distinguish system names, functions and features. Note that here the \textit{MOOD} (all capital) refers here to the name of the system network; the \textit{Mood} (first capital) is an element of clause structure formed of the Subject and Finite elements which, in fact, is not used in this work because of a general orientation towards Cardiff approach to structure; and the \textit{mood} (no capital) is a type of feature carried by finite clauses (e.g. imperative, indicative, interrogative etc.)

    Figure \ref{fig:clause-mood} presents the MOOD system network employed in the current implementation of the Parsimonious Vole parser. This MOOD system is largely identical to the one from \citet[162]{Halliday2013}. It has a few adaptations that were introduced during development of a small test corpus which is described in Chapter \ref{ch:evaluation}. The adaptations are in fact adoptions of a few traditional grammar features such as \textit{tense} and \textit{voice} but also inclusion of the \textit{agency} system \citep{Halliday2013}[350] which belongs to TRANSITIVITY system network in Sydney grammar. 
        
    
    \begin{figure}[!ht]
        \centering
        \includegraphics[width=0.8\linewidth]{Figures/SFL-grammar/clause-mood.pdf}
        \caption{An adaptation of the MOOD system network \citep[162]{Halliday2013}}
        \label{fig:clause-mood}
    \end{figure}

    The \textit{polarity} system is also adapted and was adopted from Cardiff grammar. It captures different ways in 

%    * agency system from (elsewhere)
%    * assessment extensions integrated 
      
%      SN developed during the development of the OCD corpus 
     %todo continue 
      
\subsection{TRANSITIVITY}
\label{sec:transitivity}

As explained in Section \ref{sec:functions-metafunctions} in SFL there are three metafunctions that are conflated with each other. This section briefly introduces the experiential metafunction and TRANSITIVITY as the main system that systematizes the experiential metafunction. 

In this perspective, ``the clause construes a quantum of change in a flow of events as a \textit{figure}, or a configuration of a \textit{process}, \textit{participants} involved in it and any attendant \textit{circumstance}`` \citep[212]{Halliday2013}. 

In traditional grammar the term \textit{transitivity} refers to the property of verbs according to which they are classified into transitive and intransitive. In SFL the term transitivity is primarily concerned with clauses. It is most insightful to refer to Halliday's TRANSITIVITY \citep{Halliday67-parts1+2,Halliday68-part3,Halliday68} that deals with Predicate, Subject, Complement, and Adjunct all of which are elements of the clause and are usually conflated with the Process, Participants and Circumstances. 

Sydney grammar makes a distinction between two types of experience: ``inner'' as experience inside ourselves and ``outer'' as experience in the world around us. The prototypical outer experience is that of actions and events. The inner experience is more difficult to sort but it is a kind of reply of the outer, recording it, reflecting on it, reacting on it etc. The two grammatical categories realizing the these sorts of experience are the \textit{material} process and \textit{mental} process.

In addition to material and mental there is a third kind of process used in identifying, classifying and relating various kinds of experience. The grammatical category realizing this type of links is the \textit{relational} process. 

Then using the combinations of the three main processes above, Halliday defines \textit{behavioural}, \textit{verbal} and \textit{existential} processes. 

Cardiff grammar employs similar process types, that of \textit{action}, \textit{relational}, \textit{mental} and \textit{influential}. In addition it links these process types to realms of experience: \textit{physical}, \textit{social interaction}, \textit{psychological} and \textit{abstract}.
Figure \ref{fig:cardiff-realm-processtype} provides the schematic connection between realms of experience and various process types that can realise that kind of experience \citep[37]{Fawcett2009}. 

\begin{figure}[!ht]
    \centering
    \begin{tikzpicture}[]
    \node (re) [font=\bf] {realm of experience};
    \node (ph) [below=1em of re] {physical};
    \node (si) [below=1em of ph] {social interaction};
    \node (ps) [below=1em of si] {psychological};
    \node (ab) [below=1em of ps] {physical};

    \node (pt) [font=\bf, right=7em of re] {types of processes};
    \node (ac) [below=1em of pt] {action};
    \node (re) [below=1em of ac] {relational};
    \node (me) [below=1em of re] {mental};
    \node (in) [below=1em of me] {influential};        
    
    \draw [thick] (ph.east) -- (ac.west);
    \draw [thick] (ph.east) -- (re.west);
    \draw [thick] (ph.east) -- (in.west);
    
    \draw [thick] (si.east) -- (ac.west);
    \draw [thick] (si.east) -- (re.west);
    \draw [thick] (si.east) -- (in.west);
    
    \draw [thick] (ps.east) -- (me.west);
    \draw [thick] (ps.east) -- (re.west);
    \draw [thick] (ps.east) -- (in.west);
    
    \draw [thick] (ab.east) -- (re.west);

    \end{tikzpicture}
    \caption{The connections in Cardiff grammar between realms of experience and the process types}
    \label{fig:cardiff-realm-processtype}
\end{figure}

%todo introduce Neale's work, and a few details about the processes, the participants 

\begin{figure}[!ht]
    \centering
    \includegraphics[width=0.7\linewidth]{Figures/SFL-grammar/Transitivity.pdf}
    \caption{Cardiff grammar TRANSITIVITY}
    \label{fig:cardiff-transitivity}
\end{figure}

%todo continue

\subsection{Process Type Database}
\label{sec:ptdb}
%todo

\section{Concluding remarks}
This chapter has described the grammatical units and the two system networks adopted in this work. They constitute a selection from from Sydney and Cardiff grammar implemented in the Parsimonious Vole parser.

Because of its bottom up approach to unit structure, rank scale relaxation and accommodation of embedding as a general principle, Cardiff systemic functional theory is more suitable for parsing than the Sydney one. Nonetheless the unit definitions in the Cardiff grammar are deeply semantic in nature. Parsing with such units requires most of the time lexical-semantically informed decisions beyond merely syntactic variations. This is one of the reasons why the parsing attempts by \citet{ODonoghue1991a} and others in the COMMUNAL project were all based on a corpus. It is also the reason to adapt in this thesis Sydney unit structures as they are closer to traditional grammar syntax \citep{Quirk1985}.


Next chapter lays the theoretical foundations of Dependency Grammar and introduces the Stanford dependency parser used as a departing point in current parsing pipeline. Because there is a transformation step from dependency to systemic functional consistency structure, the next chapter also covers a theoretical compatibility analysis and how such a transformation should in principle look like. 

