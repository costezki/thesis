%
% Theorem styling
%
\renewcommand{\listtheoremname}{List of definitions}
\declaretheoremstyle[spaceabove=0.75em,spacebelow=0.75em, headindent=0em, postheadspace=1em,qed={}]{definition}
\theoremstyle{definition}
\newtheorem{definition}{Definition}[section]
\newtheorem{generalization}{Generalisation}[section]
\newtheorem{proposition}{Proposition}[section]
\newtheorem{question}{Research question}%[section]
\theoremstyle{remark}
\newtheorem{remark}{Remark}


%match operator
\DeclareMathOperator{\match}{\gtrdot}
%\DeclareMathOperator{\matchedBy}{\lessdot}

% Tkiz styling
%
\depstyle{dep-style}{edge style = {thick, RoyalBlue,}, label style = {thick, text=RoyalBlue, fill=white, draw=RoyalBlue, font=\footnotesize, }, text style = {column sep=0.5cm, row sep=1ex,}, edge unit distance = 0.7em}
\depstyle{dep-style-narrow}{edge style = {thick, RoyalBlue},label style = {thick, text=RoyalBlue, fill=white, draw=RoyalBlue, font=\footnotesize,}, text style = {column sep=0.25cm, row sep=1ex}}

% Global node and Endge styling
\tikzset{
	every node/.style={
		align=center,
	},
	every path/.style={
		align=center,
	},
}

% Special styles
\tikzset{
	pattern-node/.style={
		rectangle,
		rounded corners,
		draw=black, very thick,
		minimum height=2em,
		inner sep=0.3em,
		text centered,
		align=center,
		node distance=3cm,
		anchor=center
	},
	pattern-node-negative/.style={
		rectangle,
		rounded corners,
		draw=black, very thick, dashed,
		minimum height=2em,
		inner sep=0.3em,
		text centered,
		align=center,
		node distance=3cm,
		anchor=center
	},
	edge-style/.style   = {very thick, ->,>=stealth,rounded corners=3mm },
	tree-style/.style = { %edge from parent fork down,
%		child anchor=north,
%        parent anchor=center,
        edge from parent path={(\tikzparentnode) -- (\tikzchildnode)},
        growth parent anchor=south,
        edge from parent/.style={draw,rounded corners=3mm},
		edge-style,
		level 1/.style={sibling distance=10em},
		level 2/.style={sibling distance=10em}, 
		level distance=7em,
	},
%	graph-style/.style = {sibling distance = 5em}
}

%% http://tex.stackexchange.com/questions/55068/is-there-a-tikz-equivalent-to-the-pstricks-ncbar-command
\tikzset{
	ncbar angle/.initial=90,
	ncbar/.style={
		to path=(\tikztostart)
		-- ($(\tikztostart)!#1!\pgfkeysvalueof{/tikz/ncbar angle}:(\tikztotarget)$)
		-- ($(\tikztotarget)!($(\tikztostart)!#1!\pgfkeysvalueof{/tikz/ncbar angle}:(\tikztotarget)$)!\pgfkeysvalueof{/tikz/ncbar angle}:(\tikztostart)$)
		-- (\tikztotarget)
	},
	ncbar/.default=0.5cm,
}

% Styles used for drawing in data structures chapter
\tikzset{square left brace/.style={ncbar=0.8em}}
\tikzset{square right brace/.style={ncbar=-0.8em}}
\tikzset{round left paren/.style={ncbar=0.5em,out=120,in=-120}}
\tikzset{round right paren/.style={ncbar=0.5em,out=60,in=-60}}
\tikzstyle{system-features}=[rectangle, draw=black, rounded corners, text centered, anchor=west, rectangle split, thick]
\tikzstyle{system-name}=[rectangle, draw=none, thick, text centered]
\tikzstyle{precondition} = [->, thick, black]

%style used for drawing in introduction
\tikzset{split-node/.style={%
		draw,%                    
		shape=rectangle split,%   
		rectangle split parts=3,% 
		rectangle split part align={center,center,center},%
		rectangle split part fill={blue!20,red!20,green!20}}
}

\tikzset{split-node-2/.style={%
        draw,%                  
        anchor=north,  
        shape=rectangle split,%   
        rectangle split parts=3,% 
        rectangle split part align={center,center,center},%
        rectangle split part fill={white!0,blue!20,red!20,green!20}}
}

\tikzset{split-node-21/.style={%
        draw,%                
        anchor=north,    
        shape=rectangle split,%   
        rectangle split parts=2,% 
        rectangle split part align={center,center},%
        rectangle split part fill={white!0,blue!20}}
}

\tikzset{split-node-3/.style={%
        draw,%                    
        anchor=north,
        shape=rectangle split,%   
        rectangle split parts=4,% 
        rectangle split part align={center,center,center,center},%
        rectangle split part fill={white!0,blue!20,red!20,green!20}}
}

\tikzset{split-node-31/.style={%
        draw,%                   
        anchor=north, 
        shape=rectangle split,%   
        rectangle split parts=3,% 
        rectangle split part align={center,center,center},%
        rectangle split part fill={white!0,blue!20,green!20}}
}

\tikzstyle{flow-arrow} = [signal, signal from=west thick, draw, dashed, black, rotate=270, aspect=0.10, minimum width=4em, minimum height=3.5em, align=center, fill=YellowGreen!60]

%
% some BPMN diagram propoerties
%
\tikzstyle{every task} = [thick,fill=OrangeRed!60]
\tikzstyle{every data} = [thick,fill=ForestGreen!60]
\tikzstyle{every persistent-data} = [thick,fill=BurntOrange!60]
\tikzstyle{every sequence} = [thick]
\tikzstyle{every gateway} = [thick,fill=white]
\tikzstyle{every event} = [thick,fill=white]
\tikzstyle{end event} = [event,line width=2.5pt,fill=white]
%



% algorithm2e - Set up Python like pseudo code syntax
%
\SetKwInOut{Input}{input}\SetKwInOut{Output}{output}
\SetStartEndCondition{ }{}{}%
\SetKwProg{Fn}{def}{\string:}{}
\SetKw{KwTo}{in}\SetKwFor{For}{for}{\string:}{}%
\SetKwIF{If}{ElseIf}{Else}{if}{:}{elif}{else:}{}%
\SetKwFor{While}{while}{:}{fintq}%
\SetKwComment{comment}{\#}{}
\SetKw{Raise}{raise}
\AlgoDontDisplayBlockMarkers\SetAlgoNoEnd\SetAlgoNoLine\DontPrintSemicolon%

%only these algorithm keywords shall remain
\SetKwData{dg}{dg}
\SetKwData{cg}{cg}
\SetKwData{Children}{children}
\SetKwData{stack}{constituency stack}
\SetKwData{cgPointer}{cg pointer}
\SetKwData{rrule}{rule}
\SetKwData{operation}{operation}
\SetKwData{elementType}{element type}
\SetKwData{rt}{rule table}
\SetKwData{edge}{edge}

\SetKwData{simpleKey}{generic key}
\SetKwData{ctxKey}{specific key}
\SetKwData{headPos}{head POS}
\SetKwData{tailPos}{tail POS}
\SetKwData{fn}{free nodes}
\SetKwData{node}{node}
\SetKwData{sspan}{span}
\SetKwData{Element}{element}
\SetKwData{Group}{group}
\SetKwData{whGroup}{wh-group}
\SetKwData{Function}{function}


\SetKwData{network}{network}
\SetKwData{function}{selector function}
\SetKwData{choices}{systemic choices}

\SetKwData{dict}{dictionary}
\SetKwData{patterns}{pattern set}
\SetKwData{pattern}{pattern}
\SetKwData{lexicalItem}{lexical item}
\SetKwData{feature}{feature}
\SetKwData{snet}{system network}
\SetKwData{systemm}{system}

\SetKwData{word}{word}
\SetKwData{text}{text}

\SetKwData{as}{parser segment}
\SetKwData{ms}{corpus segment}
\SetKwData{aslist}{parser segments}
\SetKwData{mslist}{corpus segments}

%\SetKwFunction{enrich}{enrich\_mcg}
%\SetKwFunction{whtrace}{create\_wh\_traces}
%\SetKwFunction{whtraceadjunct}{create\_adjunct\_wh\_traces}
%\SetKwFunction{whtracetheta}{create\_theta\_wh\_traces}
%\SetKwFunction{gencpg}{generate\_cpg\_from\_PTDB}
%\SetKwFunction{transitivity}{parse\_transitivity}
%
%%


%
% Link colors 
%
\hypersetup{
	colorlinks=true,       % false: boxed links; true: colored links
	linkcolor=RoyalBlue,          % color of internal links (change box color with linkbordercolor)
	citecolor=BrickRed,        % color of links to bibliography
	filecolor=magenta,      % color of file links
	urlcolor=RoyalBlue           % color of external links
}

%
% Bib style
%
\setcitestyle{authoryear}
\newcommand{\citets}[2][]{\citeauthor{#2}'s (\citeyear[#1]{#2})}
\bibpunct[: ]{(}{)}{;}{}{}{,}

%
% %%%%
%

\setlength{\emergencystretch}{12pt}

%
% Json Listing
%
\colorlet{punct}{red!60!black}
\definecolor{background}{HTML}{EEEEEE}
\definecolor{delim}{RGB}{20,105,176}
\colorlet{numb}{magenta!60!black}

\lstdefinelanguage{json}{
	basicstyle=\normalfont\ttfamily,
	keywordstyle=\ttfamily\bfseries,
	keywords= {false,true},
	numbers=left,
	numberstyle=\scriptsize,
	stepnumber=1,
	numbersep=8pt,
	showstringspaces=false,
	breaklines=true,
	frame=lines,
	backgroundcolor=\color{background},
	alsoletter= 0123456789.,
	literate=
	*{0}{{{\color{numb}0}}}{1}
	{1}{{{\color{numb}1}}}{1}
	{2}{{{\color{numb}2}}}{1}
	{3}{{{\color{numb}3}}}{1}
	{4}{{{\color{numb}4}}}{1}
	{5}{{{\color{numb}5}}}{1}
	{6}{{{\color{numb}6}}}{1}
	{7}{{{\color{numb}7}}}{1}
	{8}{{{\color{numb}8}}}{1}
	{9}{{{\color{numb}9}}}{1}
	{:}{{{\color{punct}{:}}}}{1}
	{,}{{{\color{punct}{,}}}}{1}
	{\{}{{{\color{delim}{\{}}}}{1}
	{\}}{{{\color{delim}{\}}}}}{1}
	{[}{{{\color{delim}{[}}}}{1}
	{]}{{{\color{delim}{]}}}}{1},
}

\lstdefinelanguage{XML}
{
    basicstyle=\ttfamily\footnotesize,
    morestring=[b]",
    moredelim=[s][\bfseries\color{Maroon}]{<}{\ },
    moredelim=[s][\bfseries\color{Maroon}]{</}{>},
    moredelim=[l][\bfseries\color{Maroon}]{/>},
    moredelim=[l][\bfseries\color{Maroon}]{>},
    morecomment=[s]{<?}{?>},
    morecomment=[s]{<!--}{-->},
    commentstyle=\color{DarkOliveGreen},
    stringstyle=\color{blue},
    identifierstyle=\color{red},
    %    copied from the JSON definition
    numbers=left,
    numberstyle=\scriptsize,
    stepnumber=1,
    numbersep=8pt,
    showstringspaces=false,
    breaklines=true,
    frame=lines,
    backgroundcolor=\color{background},    
}