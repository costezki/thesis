\chapter{Empirical evaluation}
% \label{ch:evaluation}

% Chapter Introduction
% - set up the scene
% - state what is the aim of the evaluation (accuracy of the parser to segment text, 
% assign unit classes, element functions and systemic features to text segments)
% - explain how that aim is achieved (compare the segmentation, element assignment, 
% feature assignment available in the corpus with that coming from the parser)
% - how the chapter is structured  
% Main sections
% - describe the corpora
%   (provenance, annotations, size metrics, quality assessment; data examples ?)
% - describe the method of comparing parser output to corpus annotations 
% - present the evaluation data for: Mood and Transitivity elements + unit classes 
%   (focus on constituent identification)
% - present the evaluation data for: Mood and Transitivity features 
%   (focus on feature accuracy per system for the identified units)
% - *interpret evaluation data and present findings
%   (or should this be provided when each evaluation data is presented) 
% Conclusions
% - concise summary of the main findings


    This chapter aims to evaluate the Parsimonious Vole parser accuracy at generating text analysis in general; and how well it performs at unit boundary detection (i.e text segmentation), unit class assignment, element assignment and feature selections in particular. The grammar that is employed in this evaluation was introduced in Chapter \ref{ch:the-grammar} and the corpus will be introduced in Section \ref{sec:corpus}. 
    
    The evaluation data are collected by comparing the labelled segments  available in the corpus to the labelled segments in the parser output. The main measurements of parser accuracy considered here are \textit{precision} and \textit{recall} and $F_1$ scores. The parser \textit{precision} measures how many segments have been produced by the parser that are also found in manual analysis; and the parser \textit{recall} measures how many correct segments have been produced by the parser relative to the total number of produced segments. $F_1$ score is a harmonic mean of the precision and recall.
    
    The corpus used in this evaluation do not constitute a true gold standard and there are, due to different reasons, some differences to the parser. These differences will be described in detail in Section \ref{sec:differences}.
    
    The evaluation methodology, which will be described in detail in Section \ref{sec:evaluation-methodology}, considers perfect alignment between segment boundaries and their labels, and also considers partial alignment of segments with the same label. This means that segments such as the ones in Listings \ref{lst:exampleText1} and \ref{lst:exampleText1} are given some credit in the alignment process.

\noindent
\begin{minipage}{\linewidth}
\begin{lstlisting}[numbers=left,basicstyle=\small\tt, stepnumber=1,firstnumber=0,frame=single,caption=Example segment from the corpus,label=lst:exampleText1,escapeinside={(*}{*)}]
(*\textcolor{black!50}{$_{587}$}*)forced me into treatment(*\textcolor{black!50}{$_{611}$}*)
\end{lstlisting}
\end{minipage}

\noindent
\begin{minipage}{\linewidth}
\begin{lstlisting}[numbers=left,basicstyle=\small\tt, stepnumber=1,firstnumber=0,frame=single,caption=Example segment from the parser output,label=lst:exampleText2,escapeinside={(*}{*)}]
(*\textcolor{black!50}{$_{583}$}*)and forced me into treatment .(*\textcolor{black!50}{$_{612}$}*)
\end{lstlisting}
\end{minipage}    
     

\section{Evaluation corpus}
\label{sec:corpus}
% (provenance, annotations, size metrics, quality assessment; data examples ?)

\subsection{Bremen Translation Corpus (BTC)}

    The Bremen Translation Corpus (BTC) was created at the University of Bremen by Kerstin Fischer, Anatol Stefanowitsch and Anke Schulz. It consists of comparable and parallel texts. The comparable part consists of a series of newsgroup texts of about 10,000 words of English text and another 10,000 words of German, text taken from the same register. The parallel part, called EDNA, is much larger comprising about 100,000 words of parallel English-German text. Anke uses in her thesis 10,000 words of parallel text and about the same of comparable text \citep[31]{schulz2015me}. In this evaluation only the English part is considered that comprises 31 files spanning over 1503 clauses and 20864 words. In addition, Anke provided another similar smaller corpus of 157 clauses that is also included into this evaluation. 
    
    The corpus annotations cover Cardiff TRANSITIVITY, THEME and MODIFICATION system networks. The grammatical details and the annotation methodology are covered in detail in \citet{schulz2015me}.

\subsection{Obsessive Compulsive Disorder corpus}

     The first dataset (OCD) was created by Ela Oren and myself and is focused on syntactic constituency structure and clause MOOD features. The texts represent blog articles of people diagnosed with Obsessive Compulsive Disorder (OCD) who self-report on the challenge of overcoming OCD. The dataset contains four texts comprising all together 988 clauses and 8605 words. 

\subsection{Differences between corpus and parser output}
\label{sec:differences}
% todo rename section

% todo: introduce this example
\begin{minipage}{\linewidth}
\begin{lstlisting}[numbers=left,basicstyle=\small\tt, stepnumber=1,firstnumber=0,frame=single,caption=Raw text example in annotation data,label=lst:exampleText,escapeinside={(*}{*)}]
(*\textcolor{black!50}{$_{0}$}*)Red riding hood excerpt(*\textcolor{black!50}{$_{24}$}*)
(*\textcolor{black!50}{$_{25}$}*)"What have you in that basket,   Little Red Riding Hood?"(*\textcolor{black!50}{$_{82}$}*)
(*\textcolor{black!50}{$_{83}$}*)
(*\textcolor{black!50}{$_{84}$}*)"Eggs and butter and cake, Mr. Wolf."(*\textcolor{black!50}{$_{111}$}*)
\end{lstlisting}
\end{minipage}

    Listing \ref{lst:exampleText} presents an example raw text from the annotation dataset containing an initial title line and two sentences separated by an empty line. The greyed index numbers at the beginning and end of each line indicate character offsets. In BTC corpus files, the first line plays the role of a header containing the title or the file name. In this example it is a title. Either way, this first line is neither considered for annotation nor parsing. 
    
    In the OCD file the text was not normalised before the annotation started. Mostly it is organised as one sentence per line, but there are instances of  extra blank lines or several some sentences per line as one continuous block. The text may also sporadically contain tabs and blank spaces such as here in line 1 between the comma and the ``Little Red Riding Hood''. 

    It is noteworthy to mention that there are segmentation errors in a few cases from the OCD corpus. Some segments are either shifted and include the adjacent spaces (e.g. `` getting this push'' instead of ``getting this push'') or, the converse, leave out one or two characters of a marginal word (e.g. ``the balanc'' instead of ``the balance''). Such 

    The parser diverges in a few ways from the corpus annotation methodology when it comes to punctuation marks and treatment of conjunctions. 
    
    In the corpus, punctuation marks such as commas, semicolons, three dots and full stops are not included in the constituent segments while the parser includes them at the end of each adjacent segment. 
    
    The treatment of conjunctions that was discussed in Section \ref{sec:coordination} differs as well. In the corpus, the conjunctions (such as ``and'', ``but'', ``so'', etc.) are excluded from the conjunct segments; they are considered markers in the clause/group complexes rather than part of the constituent. The parser, on the other hand, includes the conjunctions in the following adjacent segment. For example in the corpus there we find segment ``forced me into treatment'' while the parser produces a slightly larger segment ``and forced me into treatment.'' that includes the conjunction at the beginning and the full-stop at the end.
    
    Moreover the conjunct segment spans differ as well due to difference in treatment. Instead of being analysed in parallel, having sibling status as depicted in Figure \ref{fig:segment-conjunction-paralel}, the parse generated conjunct segments are subsumed in a cascade from the former to the latter as depicted in Figure \ref{fig:segment-conjunction-subsumed}.

    \begin{figure}[!ht]
        \centering
        \begin{subfigure}[b]{0.47\textwidth}
            \centering
            \begin{tikzpicture}[pattern-node]
            \node[pattern-node] (start) {};
            \node[pattern-node, right = 5em of start] (end) {};
            \draw[edge-style] (start) -- (end) node[midway, above]{conjunct 1};
            
            \node[pattern-node, right = .5em of end] (conj) {and};
            
            \node[pattern-node, right = .5em of conj] (start1) {};
            \node[pattern-node, right = 5em of start1] (end1) {};
            \draw[edge-style] (start1) -- (end1) node[midway, above]{conjunct 2};
            
            \end{tikzpicture}
            \caption{Conjuncts annotated as parallel segments}
            \label{fig:segment-conjunction-paralel}
        \end{subfigure}
        \quad
        \begin{subfigure}[b]{0.47\textwidth}
            \centering
            \begin{tikzpicture}[pattern-node] 
            \node[pattern-node] (start) {};
            \node[pattern-node, right =14em of start] (end) {};
            \draw[edge-style] (start) -- (end) node[midway, above]{conjunct 1};
            
            
            \node[pattern-node, below = 0.1em of end] (end1) {};
            \node[pattern-node, left = 8em of end1] (start1) {};
            \draw[edge-style] (start1) -- (end1) node[midway, above]{conjunct 2};
            
            \node[pattern-node, below = -0.6em of start1, xshift=1.8em] (conj) {and};        
            \end{tikzpicture}
            \caption{Conjuncts annotated as subsumed segments}
            \label{fig:segment-conjunction-subsumed}
        \end{subfigure}
        \caption{Treatment of conjunctions in the corpus compared to the parser}
        \label{fig:conjunction-treatment}
    \end{figure}

    % TODO: argue why partial match is valuable 

\section{Evaluation methodology}
\label{sec:evaluation-methodology}

\subsection{Segments}

    To compare the segment boundaries we need to understand how they are represented in each output and how they can be brought to a common form comparison. 
    All datasets were created with the UAM Corpus Tool \citep{ODonnell2008,ODonnell2008a} version 2.4. The annotations, in this software, are recorded as segments spanning from a start to an end position in the text file together with the set of features (selected from a systemic network) attributed to that segment. There are no constituency or dependency relations between segments. The XML representation of an example annotation segment is provided in Listing \ref{lst:segment1}. There the \textit{id} attribute indicates the unique identification number within the annotation dataset, the \textit{start} and \textit{end} attributes define the segment between two character offsets relative to the beginning of the text file.

\begin{minipage}{\linewidth}
\begin{lstlisting}[language=XML,basicstyle=\small\tt,frame=single,caption=Segment example in UAM corpus tool,label=lst:segment1]
<segment id="4" start="20" end="27" 
features="configuration;relational;attributive" 
state="active"/>
\end{lstlisting}
\end{minipage}

     In the current evaluation, the segments are reduced to carry only one label. The consequence is that segments with multiple features (Figure \ref{fig:segment-multiple}) are broken down into multiple segments with the same span (Figure \ref{fig:segment-simple}) for each feature. When each segment contains exactly one feature the evaluation can be focused on one or a set of features of interest by selecting only the segments that contain exactly those. 

    \begin{figure}[!ht]
        \centering
        \begin{subfigure}[b]{0.47\textwidth}
            \centering
            \begin{tikzpicture}[pattern-node]
            \node[pattern-node] (start) {20};
            \node[pattern-node, right = 7em of start] (end) {27};
            \draw[edge-style] (start) -- (end) node[midway, above]{configuration,\\relational,\\attributive};
            \end{tikzpicture}
            \caption{A segment with a set of features}
            \label{fig:segment-multiple}
        \end{subfigure}
        \begin{subfigure}[b]{0.47\textwidth}
            \centering
            \begin{tikzpicture}[pattern-node] 
            \node[pattern-node] (start1) {20};
            \node[pattern-node, right = 7em of start1] (end1) {27};
            \draw[edge-style] (start1) -- (end1) node[midway, above]{configuration};
            
            \node[pattern-node, below = 1em of start1] (start2) {20};
            \node[pattern-node, right = 7em of start2] (end2) {27};
            \draw[edge-style] (start2) -- (end2) node[midway, above]{relational};
            
            \node[pattern-node, below = 1em of start2] (start3) {20};
            \node[pattern-node, right = 7em of start3] (end3) {27};
            \draw[edge-style] (start3) -- (end3) node[midway, above]{attributive};	
            \end{tikzpicture}
            \caption{A set of segments with single features}
            \label{fig:segment-simple}
        \end{subfigure}
        \caption{Example of breaking down a segment with multiple features into set of segments with a single feature}
        \label{fig:segment-breackdown}
    \end{figure}

% todo conclude the section 

\subsection{reading the corpus segments as a set of mono labelled segments}


\subsection{Turning parser output into a set of mono labelled segments}
% todo update the section start
    
    In order to compare the parser generated output to the corpus segments they need to be turned into the same form. In this section I describe the task of turning rich constituency graphs (CG) into labelled segments similar to those in the corpus. 
    
    Once the parser receives a text as an input, it normalises and segments the text first before performing anything else. Corpus annotation is performed on the raw non-normalised text. To make the parser output segments comparable to the ones in the corpus they need to refer, in terms of their offsets and indexes, to the same raw text. Before the evaluation can take place the parser output segments need to be re-indexed to correspond to the input raw text. 
    %This sections explains the process how the output segmentation is remapped onto the original text.
    
    To fulfil this task, the text processed by the parser is re-indexed back into the original raw text at the level of words (tokens), constituents and sentences. Algorithm \ref{alg:re-index-text} provides pseudo-code of the indexing process.

    \begin{algorithm}[!ht]
        \Input {CG bundle, \text} %, \dg
        \Begin {
            offset $\leftarrow$ 0\;
            \For{\cg \KwTo CG bundle}
            {
                generate segments for \cg indexed on \text given the offset\;
                offset $\leftarrow$ the end of \cg\;
            }
        }
        \caption{Sentence level re-indexing of CG according to the raw text}
        \label{alg:re-index-text}
    \end{algorithm}

    In Section \ref{sec:creation-constituency-graph} was explained that the parser processes one sentence at the time. If more than one sentence is provided as input text the output is then a bundle of constituency graphs. The input for Algorithm \ref{alg:re-index-text} is the array of CGs produced by the parser and the original text. The result of this algorithm is a set of segments indexed according to the raw text. The task is performed by iterating the resulting constituency graphs one by one and indexing each with respect to the offset given by the previous one. The indexing of the CG structure is presented in Algorithm \ref{alg:re-index-words-and-cg}.

    \begin{algorithm}[!ht]
        \Input {\cg, \text, sentence offset} %, \dg
        \Begin {
            words $\leftarrow$ get \cg the list of words \;
            \For{\word \KwTo list of sentence word segments}
            {
                find the \word in the \text after a given sentence offset\;
                \eIf{\word found}
                {
                    start $\leftarrow$ get first word start index\;
                    end $\leftarrow$ get the last word end index\;
                    create a new segment (start, end, \word)\;                
                }
                {
                    generate a warning (manual adjustment needed)\;
                }
            }
            \For{\node \KwTo \cg in BFS postorder}
            {
                find the word span of the constituent\;
                start $\leftarrow$ get first word start index\;
                end $\leftarrow$ get the last word end index\;
                labels $\leftarrow$ get \node class, function and features\;
                create new segment (start, end, labels)\;
            }
            \Return set of segments\;
        }
        \caption{Constituent level re-indexing at the level of constituents according to the raw text}
        \label{alg:re-index-words-and-cg}
    \end{algorithm}

    The way each CG is re-indexed is described by Algorithm \ref{alg:re-index-words-and-cg}. The returned result is a set of segments from the constituency graph considering a given offset. The indexing task is performed first at the word (token) level and the corresponding segments are generated. Then for each constituent node in the CG, segments are generated based on the constituent word span which have already been re-indexed. The indexes of the constituent segments are set to be the beginning of the first word and the end of the last word. The labels assigned to the segments are the constituent unit class, function(s) and all the systemic features. As the segments can carry a single label only then for every feature, function and unit class a new segment is created. This is in line with the practice described above and contributes to clear evaluation methodology.

    Once the parser generated output is re-indexed according to the raw text and represented as a set of mono labelled segments, we can compare this output to the corpus annotations. The next section explains how this is done. 

\subsection{Alignment method and evaluation data}
    
    Both the corpus annotations and the parser output can be represented as a set of mono labelled segments on the raw corpus text. Once they are expressed in this form, we can compare the parser output to the corpus segments and evaluate its accuracy. This section explains how this comparison is done. I first present a strict method of evaluation and then introduce a permissive method of evaluation based on segment similarity. 

    First and straight forwards evaluation method is checking for a perfect match between every segment in the parser output and a segment in the corpus annotations. A perfect match would mean that given a parser segment there exists a corpus segment whose start index, end index and label are the same. 
    
    Using this method of evaluation we can count (a) how many segments with the same label match, (b) how many corpus segments are not matched and (c) how many parser segments are left unmatched. This way we get three count numbers per label for all labels used in the corpus annotation and parser output combined.
    
    In the Section \ref{sec:differences} I presented some differences between the parser output and the corpus annotations. Most of these differences are comparable especially that they manifest as slight variations in the segment spans, i.e. shifted start and/or end segment index, while the segment labels are exactly the same. 
    
    Accounting for differences in the segment spans is a well known task in the mainstream computational linguistics called \textit{text segmentation evaluation}. A variety of segmentation evaluation metrics have been proposed among which the most known are $P_k$ \citep[198--200]{beeferman1999statistical}, \textit{WindowDiff} \citep[10]{pevzner2002critique}, \textit{Segmentation Similarity} \citep[154-156]{fournier2012segmentation} and \textit{Boundary Edit Distance} \citep{fournier2013evaluating}. Each of these metrics have been shown to have some flaws: both $P_k and WindowssDiff$ under-penalise errors \citep{lamprier2007evaluation} and have a bias towards favouring segmentation with few or tightly-clustered boundaries \citep{niekrasz2010unbiased} while segmentation similarity tends to ovewrly optimistic values due to its normalisation \citep{fournier2013evaluating}. 
    
    All the above metrics take into account that the content of the segment text may vary and so they use text edit distance to account for that. In the case of the current evaluation the segments are defined on the same text and thus only the segment indexes are relevant. 
    
    Another simple metric to measure to measure the difference between the segments taking into account only their start and end indexes is that of \textit{geometric distance}. For two segments $S(start_S,end_S)$ and $T(start_T,end_T)$ the geometric distance is defined in Equation \ref{eq:distance}. We can replace the difference between start and end indexes with $\varDelta_{start}$ and $\varDelta_{end}$ notation and obtain the reduced form provided in Equation \ref{eq:distance-simpliefied}
    
    \begin{equation} \label{eq:distance}
    d= \sqrt{(start_S - start_T)^{2}+(end_S-end_T)^{2}}
    \end{equation}
    
    \begin{equation} \label{eq:distance-simpliefied}
    d= \sqrt{\varDelta_{start} ^{2}+\varDelta_{end}^{2}}
    \end{equation}

    %todo: explain which one I have selected
    Now that the metric for comparing segments have been introduced I move on to present a second more sophisticated evaluation method, which, in addition to accounting for the exact matches, accounts for close matches between segments.  
    
    Taking into account the partially matching segments brings us to the second evaluation method. In this case, the task is that of aligning two sets of labelled segments, which is almost the same as the well know problem in computer science called \textit{stable marriage problem} \citep{Gusfield1989}. I adopt onward this frame to explain the evaluation method.
    
    The standard enunciation of the stable marriage problem is provided below and is solved in an efficient algorithm named Gale-Shapley \citep{Gale1962} after its authors.
    
    \begin{quotation}
        Given \textit{n} men and \textit{n} women, where each person has ranked all members of the opposite sex in order of preference, marry the men and women together such that there are no two people of opposite sex who would both rather have each other than their current partners. When there are no such pairs of people, the set of marriages is deemed stable \citet{iwama2008}.
    %    \footnote{see \href{https://en.wikipedia.org/wiki/Stable_marriage_problem}{stable marriage problem on Wikipedia}}
    \end{quotation}

    In the context of this evaluation the group of men is associated with the segments generated automatically by the parser and the group of women with the segments available from the manual analysis. 

    The standard stable marriage problem is formulated such that there is a group of men and a group of women and each individual from each group expresses their preferences for every individual from the opposite group as an ordered list. The assumption is that the preferences of every individual are known and expressed as a complete ordered list of individuals from the opposite group ranging from the most to the least preferred one. Thus the preference list must be \textit{complete} and \textit{fully ordered}. 

    To fulfil these requirements I construct a distance matrix from each automatically created segment to every manually created one. The distance measure considered here is the Euclidean one provided in Equation \ref{eq:distance-simpliefied} above. The matrix represents the complete and fully ordered set of preferences stipulated in the original problem formulation. In addition to having identical offsets, the segments need to carry the same labels in order to be considered a match. This condition is not expressed in the original problem but is considered in Algorithm \ref{alg:matching}. 

    \begin{algorithm}[!ht]
        \Input{\aslist, \mslist} %, \dg
        \Begin{
            mark all \aslist and \mslist free\;
            compute distances from each \mslist to every \aslist\;
            \While{$\exists$ free \aslist}{
                \as $\leftarrow$ first free from \aslist\;
               \If{$\exists$ \mslist not yet tested to match \as}{
                   \ms $\leftarrow$ the nearest among \mslist to \as with identical label \;
                   \If{\ms is free}{
                       match \as and \ms \;
                       mark \as and \ms as non-free \;
                   }               
               \Else{
                       $\as'$ $\leftarrow$ the current match of \ms \;
                       \If{\as is closer to \ms than $\as'$}{
                           match \as and \ms \;
                           mark \as and \ms as non-free \;
                           mark $\as'$ as free \; 
                       }
                   }
               }
               \Else{
                   mark \as as non-free and non-matching \;
               }
            }
        }
        \caption{The algorithm for matching automatic and manual segments}
        \label{alg:matching}
    \end{algorithm}

    Using the second method of evaluation, presented above, we can count for every distinct label (a) how many segments match perfectly, i.e. the distance is zero, (b) how many segments partially match, i.e. the distance is greater than zero, (c) how many corpus segments are not matched and (d) how many parser segments are left unmatched. This way we get a four count numbers per label for all labels used in the corpus annotation and parser output combined that we can use to compute parser accuracy. Further more the partial matches can be analysed to estimate the degree of the deviation and derive insights what can be done about it. Having said that I further proceed with presenting the evaluation data. 

\section{Evaluation data}



