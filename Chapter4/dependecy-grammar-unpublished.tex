\subsection{Old Dependecy grammar introduction}
Traditionally, Latin language tended to be analysed with the \textit{dependency model} based on theories of what was called \textit{government}. It explains the syntagmatic relations of the subtype Firth called \textit{colligation} (i.e. relations between grammatical categories). Government is a way of explaining the rich inflections of language (such as Latin) in terms of how particular words govern, that is to say, determine, the inflection of other words. \citep[p.66]{McDonald2008}

\citet{Tesniere2015} explains how government works by giving example of Latin text analysis where the inflections immediately give aloto of information about relations between different words. For example the verb agree in person and number with its subject. From this point of view the verb governs the subject. The verb also governs complements and adjuncts which can be seen as relations of dependency between a governing element or controller and a governed element or dependant. It is important to note that Tesniere analysed syntax at the level of clause where he identified a verb node as the main controller. 

Contrary to Latin, languages like English or Chinese, where there is little or most of the time none of the inflectional marking to identify dependency relations, are much harder to analyse in terms of such relations. This was a motivation for Tesniere to reinterpret dependency relation in semantic terms rather than inflectional marking. As \citet{McDonald2008} points out, this can be regarded as extending syntagmatic relations of the clause to include what Firth was calling \textit{collocation} relations (i.e. links between lexical items). Tesniere framed his theory in terms of syntagmatic relations as expressing a model of experience. He compares the verbal node of the clause to a complete little drama. Like a drama, it obligatorily consists of an action, most often actors and features of settings. Expressed in terms of syntactic structure, the action, actors and settings become the verb, participants and circumstances \citep{Tesniere2015}. 

Further, Tesniere explains \textit{categories of language} as \textit{categories of thought}. The human mind shapes the world in its own measure by organising experience into a systematic framework of ideas and beliefs called categories of though. Likewise, the language shapes thought in its own measure by organising it into a systematic framework of grammatical categories \citep{Tesniere2015}. He stresses though, that the latter ones can vary considerably from language to language and that the analysis of syntactic relations shall be carried on not in terms of grammatical categories but rather in terms of functions. He explain through an example that analysis in terms of nouns and verbs i.e. grammatical categories, tells nothing about the tie that links the words, whereas if we turn to notions such as subject and complement it all of the sudden becomes clear: the connections are established, the lifeless words become a living organism and the sentence take on a meaning \citep{Tesniere2015}.