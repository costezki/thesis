\chapter{Mood parsing: the syntax}
\label{ch:parsing-algorithm}

%\section{Syntactic parsing}
%    \subsection{Stage 1: Preprocessing} 
%    \subsection{Stage 2: The Implemented System Networks}
%    \subsection{CG generation}
%    \subsection{CG enrichment} 
%
%\section{Semantic Parsing}
%\subsection{generation of empty elements}
%\subsection{PTDB}
%\subsection{generation of patterns from PTDB}
%\subsection{enrichment of CG graph from PTDB patetrns}

%The parsing algorithm transforms DGs into rich SF constituency graphs.
%
%* pre-process
%
%* 1st Mapping - CG with multiple class assignments - top down approach
%
%%* Class Reduction - Revised internal structure and reduced class assignments - bottom up
%
%* Enrichment of units with systemic feature selections 
%
%* Null element creation
%
%* Transitivity parsing
%
%- Generating Transitivity patterns from PTDB
%
%- Enrichment with TR Patterns

\section{Algorithm overview}


\section{Preprocessing -- canonicalization of DGs}
\label{sec:preprocessing1}
The Stanford Parser applies various machine learning(ML) techniques to parsing. It's accuracy increased over time to $\approx92\%$ for unlabelled attachments and $\approx89\%$ for labeled ones (in the version 3.5.1). This section addresses known error classes of wrongly attached nodes or wrongly labelled edges and nodes. 

As the Stanford parser evolved, some error classes changed from one version to another (v2.0.3 -- v3.2.0 -- 3.5.1). Also the set of dependency labels for English initially described in \citep{Marneffe2008, Marneffe2008a} changed to a cross-linguistic one (starting from v3.3.0) described in \citep{Marneffe2014}. 

Beside stable errors, there are two other phenomena that are modified in the preprocessing phase: \textit{copula} and \textit{coordination}. They are not errors per se but simply an incompatibility between how Stanford parser represents them and how they need to be represented for processing by the current algorithm and grammar.

In this section I describe a set of transformation operations on the dependency graph before it is transformed into systemic constituency graph. The role of preprocessing phase is bringing in line aspects of dependency parse to a form compatible with systemic constituency graph creation process by (a) correcting known errors in DG, (b) cutting down some DG edges to form a tree (c) changing Stanford parser's standard handling of copulas, coordination and few other phenomena. This is achieved via three transformation types: (a) \textit{relabelling of edge relations}, (b) relabelling node POS, and (c) reattachment of nodes to a different parent. 

\subsection{Loosening conjunction edges}
Stanford parser employs an extra edge for each of the conjuncts such that there is one indicating the syntactic relationship to the child or parent nodes (just like for any other nodes) and additionally one that shows the conjunction relationship to its sibling nodes. This process removes the parent or child relations except for the first conjunct and leaves only the sibling relationships. 

Some common patterns occurring between noun, verb and adjective conjuncts are depicted below in figures \ref{fig:conj-noun-subj} - \ref{fig:conj-verb-subj}.

\begin{figure}[H]
\centering
\begin{minipage}[b]{0.45\textwidth}
\centering
	\begin{dependency}[dep-style]
		\begin{deptext}[]
	Cats \& and \& dogs \& can \& be \& friends \& . \\ 
		\end{deptext}
	%	\deproot{3}{root}
	%	\depedge{3}{1}{aux}
	%	\depedge{3}{2}{nsubj}
	%	\depedge{3}{4}{iobj}
	%	\depedge{6}{5}{det}
		\depedge{6}{1}{nsubj}
		\depedge{6}{3}{nsubj}	
		\depedge{1}{3}{conj\_and}
	\end{dependency}
\caption{Conjunction of noun objects}
\label{fig:conj-noun-subj}
\end{minipage}
\quad
\begin{minipage}[b]{0.45\textwidth}
\centering
	\begin{dependency}[dep-style]
		\begin{deptext}[]
	Please \& give \& me \& a \& fork \& and \& a \& knife \& ? \\ % \& , \& makes \& ... \\
		\end{deptext}
	%	\deproot{3}{root}
	%	\depedge{3}{1}{aux}
	%	\depedge{3}{2}{nsubj}
	%	\depedge{3}{4}{iobj}
	%	\depedge{6}{5}{det}
		\depedge{3}{6}{dobj}
		\depedge{3}{8}{dobj}	
		\depedge{6}{8}{conj\_and}
	\end{dependency}
\caption{Conjunction of noun objects}
\label{fig:conj-noun-obj}
\end{minipage}
\end{figure}

\begin{figure}[H]
\centering
\begin{minipage}[b]{0.45\textwidth}
\centering
\begin{dependency}[dep-style]
		\begin{deptext}[]
	It \& is \& hard \& for \& both \& me \& and \& you \& . \\ % \& , \& makes \& ... \\
		\end{deptext}
		%\deproot{3}{root}
		%\depedge{3}{1}{expl}
		%\depedge{3}{2}{cop}
		%\depedge{6}{5}{preconj}
		\depedge{3}{6}{prep\_for}
		\depedge{3}{8}{prep\_for}
		\depedge{6}{8}{conj\_and}
	\end{dependency}
\caption{Conjunction of prepositional phrases}
\label{fig:conj-preps}
\end{minipage}
\quad
\begin{minipage}[b]{0.45\textwidth}
\centering
	\begin{dependency}[dep-style]
		\begin{deptext}[]
	He \& is \& strong \& and \& brave \& . \\
		\end{deptext}
		%\deproot{3}{root}
		\depedge{3}{1}{nsubj}
		\depedge{5}{1}{nsubj}
		%\depedge{3}{2}{cop}
		\depedge{3}{5}{conj\_and}
	\end{dependency}
\caption{Conjunction of copulatives sharing the subject}
\label{fig:conj-copula-subj}
\end{minipage}
\end{figure}
\begin{figure}[!ht]
\centering
	\begin{dependency}[dep-style]
		\begin{deptext}[]
	He \& came \& home \& and \& immediately \& washed \& his \& hands \& . \\ % \& , \& makes \& ... \\
		\end{deptext}
		%\deproot{2}{root}
		\depedge{2}{1}{nsubj}
		\depedge{6}{1}{nsubj}
		%\depedge{2}{3}{dobj}
		\depedge{2}{6}{conj\_and}
		%\depedge{6}{5}{advmod}
		%\depedge{8}{7}{poss}
		%\depedge{6}{8}{dobj}
	\end{dependency}
\caption{Conjunction of verbs sharing the same subject}
\label{fig:conj-verb-subj}
\end{figure}

The main reason these extra edges need to be removed is to avoid double traversal of the same node via different paths (which will lead to creation of two constituents).  For example, if multiple subject relations occur in the DG then multiple subject are going to be instantiated in CG which is not intended in the grammar. Rather only one complex unit needs to be created with the subject role composed of two noun phrases (see discussion in the Section \ref{sec:coordination}). 

The straight forward way to fix it this problem is removing functional edges to/from each conjunct except the first one. There are two generic patterns in figures \ref{fig:conj-inc-tight} and \ref{fig:conj-out-tight} correspondingly with incoming and outgoing edges that are transformed into the forms depicted in \ref{fig:conj-inc-loose} and \ref{fig:conj-out-loose}. 

I split the cases into two: patterns with incoming dependency edges and outgoing ones. First, see the pattern of conjuncts with \textit{incoming dependency} relations represented in Figure \ref{fig:conj-inc-tight} and exemplified in Figures \ref{fig:conj-noun-subj} - \ref{fig:conj-preps}. In SFG terms it corresponds to cases when the functional element of a parent constituent is filled by a complex unit below.

\begin{figure}[H]
\begin{minipage}{0.45\linewidth}
	\begin{tikzpicture}[]
	\node[pattern-node, anchor=center] (e){X};
	\node[pattern-node, anchor=center, below left =5em of e] (c1){Conj_0};
	\node[pattern-node, anchor=center, below  =3.5em of e] (c){Conj_{...}};
	\node[pattern-node, anchor=center, below right =5em of e] (c2){Conj_n};
	\path (e) edge[edge-style] node[left] {rel_1} (c1);
	\path (e) edge[edge-style] node[left] {rel_1} (c);
	\path (e) edge[edge-style] node[right] {rel_1} (c2);
	\path (c1) edge[edge-style] node[above] {conj} (c);
	\path (c) edge[edge-style] node[above] {conj} (c2);
	\end{tikzpicture}
\caption{Conjuncted elements with incoming tightly connected dependencies}
\label{fig:conj-inc-tight}
\end{minipage}
\quad
\begin{minipage}{0.45\linewidth}
	\begin{tikzpicture}[]
	\node[pattern-node, anchor=center] (e){X};
	\node[pattern-node, anchor=center, below left =5em of e] (c1){Conj_0};
	\node[pattern-node, anchor=center, below  =3.5em of e] (c){Conj_{...}};
	\node[pattern-node, anchor=center, below right =5em of e] (c2){Conj_n};
	\path (e) edge[edge-style] node[left] {rel_1} (c1);
	%\path (e) edge[edge-style] node[left] {rel_1} (c);
	%\path (e) edge[edge-style] node[right] {rel_1} (c2);
	\path (c1) edge[edge-style] node[above] {conj} (c);
	\path (c) edge[edge-style] node[above] {conj} (c2);
	\end{tikzpicture}
	\caption{Conjuncted elements with incoming loosely connected dependencies}
	\label{fig:conj-inc-loose}
\end{minipage}
\end{figure}

The second is the pattern of conjuncts with \textit{outgoing dependency relations} depicted in Figure \ref{fig:conj-out-tight}. In SFG terms it correspond to cases when a unit is sharing an element with another conjunct unit. These are mainly the cases of conjuncted verbs or copulas and are further discussed in the Chapter \ref{ch:gbt} about null elements. In GBT terms, the second to last conjuncts may miss for example the subject constituent if the conjuncts are verbs or copulas as in Figures  \ref{fig:conj-copula-subj} - \ref{fig:conj-verb-subj}. 

\begin{figure}[H]
\begin{minipage}{0.45\linewidth}
	\begin{tikzpicture}[]
	\node[pattern-node, anchor=center] (e){X};
	\node[pattern-node, anchor=center, below left =5em of e] (c1){Conj_0};
	\node[pattern-node, anchor=center, below  =3.5em of e] (c){Conj_{...}};
	\node[pattern-node, anchor=center, below right =5em of e] (c2){Conj_n};
	\path (c1) edge[edge-style] node[left] {rel_1} (e);
	\path (c) edge[edge-style] node[left] {rel_1} (e);
	\path (c2) edge[edge-style] node[right] {rel_1} (e);
	\path (c1) edge[edge-style] node[above] {conj} (c);
	\path (c) edge[edge-style] node[above] {conj} (c2);
	\end{tikzpicture}
\caption{Conjuncted elements with outgoing tightly connected dependencies}
\label{fig:conj-out-tight}
\end{minipage}
\quad
\begin{minipage}{0.45\linewidth}
	\begin{tikzpicture}[]
	\node[pattern-node, anchor=center] (e){X};
	\node[pattern-node, anchor=center, below left =5em of e] (c1){Conj_0};
	\node[pattern-node, anchor=center, below  =3.5em of e] (c){Conj_{...}};
	\node[pattern-node, anchor=center, below right =5em of e] (c2){Conj_n};
	\path (c1) edge[edge-style] node[left] {rel_1} (e);
	%\path (e) edge[edge-style] node[left] {rel_1} (c);
	%\path (e) edge[edge-style] node[right] {rel_1} (c2);
	\path (c1) edge[edge-style] node[above] {conj} (c);
	\path (c) edge[edge-style] node[above] {conj} (c2);
	\end{tikzpicture}
	\caption{Conjuncted elements with outgoing loosely connected dependencies}
	\label{fig:conj-out-loose}
\end{minipage}
\end{figure}

\subsection{Transforming copulas into verb centred clauses}
\label{sec:copulas}
In Stanford dependency grammar \textit{copular verbs} are treated as dependants of their complements (see Figures \ref{fig:copula-simple} and \ref{fig:copula-conj}) because of the intention to maximize connections between content words. This configuration breaks the rule of the main verb being the head of clause discussed in Sections \ref{sec:verbal-grpoup-and-clause-division} and \ref{sec:cardiff-clause}. 

Moreover, despite that a variety of verbs are recognised as copulative e.g. \textit{act, keep, sound}, etc. Stanford parser provides copula configurations only for the verb \textit{to be} leading to unequal treatment of copular verbs. 

This case is sometimes accompanied by two relations that create cycles in th DG. They are the \textit{xsubj}, the relation to a controlling subject and \textit{ref}, the relation to a referent. The two relations are removed and their resolution is transferred to the semantic analysis stage of the algorithm.

\begin{figure}[H]
\centering
\begin{minipage}[b]{0.45\textwidth}
\centering
\begin{dependency}[dep-style]
		\begin{deptext}[]
	She \& is \& beautiful \& in \& the \& moonlight \&. \\
		\end{deptext}
		\deproot{3}{root}
		\depedge{3}{2}{cop}
		\depedge{3}{1}{nsubj}
		\depedge{3}{6}{prep\_in}
		\depedge{6}{5}{det}
	\end{dependency}
\caption{Conjunction of prepositional phrases}
\label{fig:copula-simple}
\end{minipage}
\quad
\begin{minipage}[b]{0.45\textwidth}
\centering
	\begin{dependency}[dep-style]
		\begin{deptext}[]
	He \& is \& strong \& and \& brave \& . \\
		\end{deptext}
		\deproot{3}{root}
		\depedge{3}{1}{nsubj}
		\depedge{5}{1}{nsubj}
		\depedge{3}{2}{cop}
		\depedge{3}{5}{conj\_and}
	\end{dependency}
\caption{Conjunction of copulatives sharing the subject}
\label{fig:copula-conj}
\end{minipage}
\end{figure}

To make the copulative verb the roots of its clause, following rules are implemented. First, some relations are transferred from the copula complement (adjective JJ or noun NN) to the copulative verb. The transferred relations are listed in Table \ref{tab:rependent-relations} which distinguishes them based on the part of speech which of the \textit{copula complement}.

\begin{table}[!ht]
\centering
\begin{tabulary}{\textwidth}{|c|C|}
\hline \textit{part of speech} & \textit{dominated relation} \\ 
\hline NN & dep, poss, possesive,
                 amod, appos, conj,
                 mwe, infmod, nn, num,
                 number, partmod, preconj,
                 predet, quantmod, rcmod,ref, det\\ 
\hline JJ & advmod, amod, conj \\ 
\hline 
\end{tabulary}
\caption{Relations dependent on the POS of the dominant node}
\label{tab:rependent-relations}
\end{table}

Second, all the outgoing connections from the copula complement are transferred to the verb except those listed in second column of table \ref{tab:rependent-relations}, these relations must stay linked to the NN or JJ nodes. Third the \textit{cop} relation is deleted. Fourth, all the incoming relations to the copula complement are transferred indistinguishably to the verb because these are all clause related and shall be linked to the clause dominant node. Finally the \textit{dobj} link is created from the verb to the complement noun/adjective.

Figure \ref{fig:copula-initial} represents the generic pattern of copulas in Stanford DGs. The outgoing relations are distinguished between those in the filter as \textit{rel\_dep} and the rest simply as \textit{rel} while the incoming relations are not discriminated. Figure \ref{fig:copula-final} captures the final state of the transformation where the filtered outgoing relations stay attached to the complement node while the rest incoming and outgoing relations are moved to the verb.  

\begin{figure}[H]
\begin{minipage}{0.45\linewidth}
	\begin{tikzpicture}[]
	\node[pattern-node, anchor=center] (cop){VB};
	\node[pattern-node, anchor=center, right =5em of cop] (compl){JJ/NN};
	\node[, anchor=center, above left=5em of cop,xshift=3.5em] (c1){};
	\node[, anchor=center, above right=5em of cop,xshift=-3.5em] (c2){};
	\node[, anchor=center, above left=5em of compl,xshift=3.5em] (c3){};
	\node[, anchor=center, above =3.5em of compl,] (c4){};
	\node[, anchor=center, above right=5em of compl,xshift=-3.5em] (c5){};
	\path (compl) edge[edge-style] node[above] {cop} (cop);
	\path (compl) edge[edge-style] node[right] {rel\_dep_{0..n}} (c5);
	\path (compl) edge[edge-style] node[left] {rel_{0..n}} (c2);
	\path (c4) edge[edge-style] node[above left] {rel_{0..n}} (compl);
	\end{tikzpicture}
\caption{Generic pattern for copulas in Stanford parser.}
\label{fig:copula-initial}
\end{minipage}
\quad
\begin{minipage}{0.45\linewidth}
	\begin{tikzpicture}[]
	\node[pattern-node, anchor=center] (cop){VB};
	\node[pattern-node, anchor=center, right =5em of cop] (compl){JJ/NN};
	\node[, anchor=center, above left=5em of cop,xshift=3.5em] (c1){};
	\node[, anchor=center, above right=5em of cop,xshift=-3.5em] (c2){};
	\node[, anchor=center, above left=5em of compl,xshift=3.5em] (c3){};
	\node[, anchor=center, above =3.5em of compl,] (c4){};
	\node[, anchor=center, above right=5em of compl,xshift=-3.5em] (c5){};
	\path (cop) edge[edge-style] node[above] {dobj} (compl);
	\path (compl) edge[edge-style] node[right] {rel\_dep_{0..n}} (c5);
	\path (cop) edge[edge-style] node[right] {rel_{0..n}} (c2);
	\path (c1) edge[edge-style] node[left] {rel_{0..n}} (cop);
	\end{tikzpicture}
	\caption{Generic pattern for copulas after the transformation (the same as non-copular verbs).}
	\label{fig:copula-final}
\end{minipage}
\end{figure}

In case of conjuncted copulas like in the example in Figure \ref{fig:copula-conj} the approach is slightly complicated by the fact that copula resolution algorithm shall be executed for each copula conjunct, however because of the previous step which is loosening the conjunction and removing graph cycles then only the first copula conjunct is concerned.

\subsection{Non-finite clausal complements with adjectival predicates (a pseudo-copula pattern)}
\label{sec:nonfinite-clausal-complement}
The Figure \ref{fig:clausal-complement-jj} represents a dependency parse exemplifying a clausal complement with an adjectival predicate. In this analysis there is a main clause governed by the verb \textit{to paint} and the second one by the adjective \textit{white}. 
In SFL Figure \ref{fig:clausal-complement-jj} receives a different analysis as it is represented in Table \ref{tab:adjectival-complement}.

\begin{figure}[H]
	\centering
	\begin{dependency}[dep-style]
		\begin{deptext}[]
			Perhaps \&Sarah \&painted \&the \&wall \&white \&. \\
		\end{deptext}
		\deproot{3}{root}
		\depedge{3}{1}{advmod}
		\depedge{3}{2}{nsubj}
		\depedge{5}{4}{det}
		\depedge{6}{5}{nsubj}
		\depedge{3}{6}{xcomp}
	\end{dependency}
	\caption{Dependency parse for clausal complement with adjectival predicate}
	\label{fig:clausal-complement-jj}
\end{figure} 

\begin{table}[H]
	\centering
	\begin{tabular}{|c|c|c|c|c|c|}
		\hline
		\textit{Perhaps} & \textit{Sarah} & \textit{painted} & \textit{the}   & \textit{wall}  & \textit{white.} \\ \hline
		Adjunct          & Subject        & Finite/Main Verb & \multicolumn{2}{c|}{Complement} & Complement      \\ \hline
		& Agent          & Material Action  & \multicolumn{2}{c|}{Affected}   & Attribute       \\ \hline
	\end{tabular}
	\caption{SFG analysis with attributive adjectival complement}
	\label{tab:adjectival-complement}
\end{table}

\textit{xcomp} relation defined in \citep{Marneffe2008} to introduce non-finite clausal complement without a subject. Dependency grammar allows adjectives (JJ) and nouns (NN) to be heads of clauses but only when they are a part of a copulative construction. In figure \ref{fig:clausal-complement-jj} it is not the case, there is no copulative verb \textit{to be} and also \textit{the wall} receives the subject role in the complement clause which should be absent. 

So I treat it as a misuse of \textit{xcomp} relation and the adjective should not be treated as governing a new clause but rather non-clausally complementing the verb \textit{to paint}. Moreover that in SFG adjectival predicates are not allowed.

Certainly, depending on the linguistic school, opinions may diverge on the syntactic analysis comprising one or two clause. But when analysed from a semantic perspective it is hard to deny that there is a Material Process with an Agent and Affected thing which is specifying also the resultant (or goal) Attribute of the Affected thing. 

To accommodate such cases the dependency graph is changed from pattern in Figure \ref{fig:xcomp-jj-init} to form in Figure \ref{fig:xcomp-jj-final}. The \textit{xcomp} relation is transformed into \textit{dobj} and the subject of the embedded clause (if any) becomes the direct object (\textit{dobj}) in the main clause.

\begin{figure}[H]
	\begin{minipage}{0.45\linewidth}
		\centering
		\begin{tikzpicture}[]
		\node[pattern-node, anchor=center] (vb){VB};
		\node[pattern-node, anchor=center, below right=2em of vb] (nsubj){NN};
		\node[pattern-node, anchor=center, right =3em of nsubj] (xcomp){JJ};
		
		\path (vb) edge[edge-style] node[above] {xcomp} (xcomp);
		\path (xcomp) edge[edge-style] node[below] {nsubj} (nsubj);
		\end{tikzpicture}
		\caption{Adjectival clausal complement}
		\label{fig:xcomp-jj-init}
	\end{minipage}
	\quad
	\begin{minipage}{0.45\linewidth}
		\centering
		\begin{tikzpicture}[]
			\node[pattern-node, anchor=center] (vb){VB};
			\node[pattern-node, anchor=center, below right=2em of vb] (nsubj){NN};
			\node[pattern-node, anchor=center, right =3em of nsubj] (xcomp){JJ};
		\path (vb) edge[edge-style] node[right] {dobj} (xcomp);
		\path (vb) edge[edge-style] node[below left] {dobj} (nsubj);
		\end{tikzpicture}
		\caption{Adjectival clausal complement as secondary direct object}
		\label{fig:xcomp-jj-final}
	\end{minipage}
\end{figure}

\section{Preprocessing -- error correction}
%TODO group errors by operation type
As noted by \citet{Cer2010a} the most frequent errors are related to structures that are hard to attach i.e. prepositional phrases and relative clauses. During the implementation of current parser there had been discovered a set of errors, most frequent of which are described in this section and how are they treated. These errors are specific to Stanford Parser versions v2.0.3 -- 3.2.0. This section may constitute a valuable feedback for SDP error analysis. 

\subsection{\textit{prep} relations from verb to free preposition}
As noted before only the collapsed version of the DGs are taken as input. This means that no pure \textit{prep} relations shall occur but their expended version with the specific preposition appended to the relation name i.e. \textit{prep\_xxx}.

This is not always the case especially with phrasal verbs, the \textit{prt} relations are mislabelled as \textit{prep}. The correction consist in changing the \textit{prep} (figure \ref{fig:prep-init}) into \textit{prt} (figure \ref{fig:prep-final}) if the preposition node has no children e.g. \textit{pobj}. 

\begin{figure}[!ht]
\begin{minipage}{0.45\linewidth}
\centering
	\begin{tikzpicture}[]
	\node[pattern-node, anchor=center] (vb){VB};
	\node[pattern-node, anchor=center, right =5em of cop] (pr){IN};
	\path (vb) edge[edge-style] node[above] {prep} (pr);
	\end{tikzpicture}
\caption{Mislabelled relation to free preposition.}
\label{fig:prep-init}
\end{minipage}
\quad
\begin{minipage}{0.45\linewidth}
\centering
	\begin{tikzpicture}[]
	\node[pattern-node, anchor=center] (vb){VB};
	\node[pattern-node, anchor=center, right =5em of cop] (pr){IN};
	\path (vb) edge[edge-style] node[above] {prt} (pr);
	\end{tikzpicture}
	\caption{Corrected relation to free preposition as verbal particle}
	\label{fig:prep-final}
\end{minipage}
\end{figure}

\subsection{Non-finite clausal complements with internal subjects}
The \textit{xcomp} relation stands for open clausal complements of either a verb(VB) or adjective (JJ/ADJP). The latter is actually transformed as discussed in Section \ref{sec:nonfinite-clausal-complement}. The open clausal complement defined in Lexical Functional Grammar \cite[p270--275]{Bresnan2001} is always non-finite and does not have its own subject. However sometimes \textit{xcomp} relation appears either (a) with finite verbs or (b) with own local subjects and both cases correspond to definition of \textit{ccomp} relation. 

To fix this I transform all the instances of \textit{xcomp} relation to \textit{ccomp} if the dependent verb has a local subject (nsubj) or a finite verb as depicted in Figures \ref{fig:xcomp-init} - \ref{fig:xcomp-final}.

\begin{figure}[H]
\begin{minipage}{0.45\linewidth}
\centering
	\begin{tikzpicture}[]
	\node[pattern-node, anchor=center] (vb){VB1};
	\node[pattern-node, anchor=center, below =3.5em of vb] (subj){NN};
	\node[pattern-node, anchor=center, below right=5em of vb] (comp){VB2};
	\path (comp) edge[edge-style] node[above] {nsubj} (subj);
	\path (vb) edge[edge-style] node[above right] {xcomp} (comp);
	\end{tikzpicture}
\caption{Mislabelled clausal complement}
\label{fig:xcomp-init}
\end{minipage}
\quad
\begin{minipage}{0.45\linewidth}
\centering
	\begin{tikzpicture}[]
	\node[pattern-node, anchor=center] (vb){VB1};
		\node[pattern-node, anchor=center, below =3.5em of vb] (subj){NN};
		\node[pattern-node, anchor=center, below right=5em of vb] (comp){VB2};
		\path (comp) edge[edge-style] node[above] {nsubj} (subj);
		\path (vb) edge[edge-style] node[above right] {ccomp} (comp);
	\end{tikzpicture}
	\caption{Corrected clausal complement}
	\label{fig:xcomp-final}
\end{minipage}
\end{figure}

%2. xcomp with subject  # if (VB-xcomp-VB1, VB1-[csubj,nsubj,nsubjpass]) then transform into VB-ccompl-VB1 
%	it seems possible in SD but we don't want it. 
%	    for a clause to be xcomped and have a subject. 
%	     even if the xcomped clause is usually nonfinite, it still has subject.
%	     ccomped clauses are expected to be finite and have subject.
%	     we rurn the xcomps with subj into ccomps with subj, even if they 
%	     are nonfinite
    
\subsection{The first auxiliary with non-finite POS}
%4. correct_pos_of_1st_auxiliary(result):
%    """ if the predicate has several aux verbs, the 1st one is finite, 
%    hence shall have correct POS ["VBD"] = past, ["VBP","VBZ"]= present, ["MD"]- modal 
Sometimes the first auxiliary in a clause is mistakenly labelled as a non-finite verb. For some words the exact POS is less important as it has not big impact on the CG graph and features but in the case of first auxiliary verb of a clause it makes a big difference. It has an impact on determining the finiteness of the clause in a latter stage of the algorithm.

The algorithm is thus checking that the POS of the first auxiliary is according to the mapping defined in the Table \ref{tab:aux-pos}.
\begin{table}[!ht]
\centering
	\begin{tabulary}{0.8\textwidth}{|L|L|L|}
	\hline \textit{word} & \textit{POS} & \textit{notes} \\ 
	\hline shall, should, must, may, might, can, could, will, would & MD & modals \\ 
	\hline do, have, am, are & VBP & present \\ 
	\hline has, is & VBZ & present 3rd person \\ 
	\hline did, had, was, were & VBD & past \\ 
	\hline 
	\end{tabulary}
\caption{Mapping lexical forms of auxiliaries to their POS}
\label{tab:aux-pos}
\end{table}

\subsection{Prepositional phrases as false prepositional clauses}
%5. correct_pepc_without_verb(result):
%    """ 
%        assuming that the copulas are already changed,
%      the successor node of the prepc link shall be a verb,
%       otherwise it is not a prepc but a prep link
%    """

\textit{prepc} is a relation that introduces, via a preposition, a clausal modifier for a verb, adjective or noun. Assuming that the copulas had been changed as described in subsection \ref{sec:copulas} then the head and the tail of the relation can only be a verb. However when the relation head is not a verb (only nouns encountered so far) then the relation needs to be corrected from \textit{prepc} to \textit{prep} introducing a prepositional phrase rather than a subordinate clause. 

\begin{figure}[!ht]
\begin{minipage}{0.45\linewidth}
\centering
	\begin{tikzpicture}[]
	\node[pattern-node, anchor=center] (vb1){VB1};
	\node[pattern-node, anchor=center, right=3em of vb1, yshift=-2em] (in){IN};
	\node[pattern-node, anchor=center, right=3em of in, yshift=2em] (vb2){NN};
	\path (vb1) edge[edge-style] node[above] {prepc\_xxx} (vb2);
	\end{tikzpicture}
\caption{Mislabelled prepositional phrase as clausal modifier}
\label{fig:prepc-init}
\end{minipage}
\quad
\begin{minipage}{0.45\linewidth}
\centering
	\begin{tikzpicture}[]
	\node[pattern-node, anchor=center] (vb1){VB1};
	\node[pattern-node, anchor=center, right=3em of vb1, yshift=-2em] (in){IN};
	\node[pattern-node, anchor=center, right=3em of in, yshift=2em] (vb2){NN};
	\path (vb1) edge[edge-style] node[above] {prep\_xxx} (vb2);
	\end{tikzpicture}
	\caption{Corrected prepositional phrase}
	\label{fig:prepc-final}
\end{minipage}
\end{figure}

\subsection{Mislabelled infinitives}
%6. correct_imperative_non_infinitives(result):
%    """ 
%    if there is a VB pos that does not have an 'aux' dependecy  to preposition 'to'
%     then it is rather present simple than infinitive so the verb POs is VBP 
%    """
In English base form of the verb often coincides with present simple form (non 3^{rd} person). Therefore the POS tagger sometimes mislabels infinitive (VB) as present simple (VBP) the verb is  and vice versa. 


The algorithm checks the presence of the preposition \textit{to} (linked via \textit{aux} dependency relation) in front of the verb. If the preposition is present then the verb POS is changed to VB and reverse, if the auxiliary preposition is not present the verb POS is changed into VBP. 

\begin{figure}[H]
\begin{minipage}{0.45\linewidth}
\centering
	\begin{tikzpicture}[]
	\node[pattern-node, anchor=center] (vb1){VBP};
	\node[pattern-node, anchor=center, left=3em of vb1, ] (in){TO};
	\path (vb1) edge[edge-style] node[above] {aux} (in);
	\end{tikzpicture}
\caption{Infinitive mislabelled as present simple}
\label{fig:infinitive-init}
\end{minipage}
\quad
\begin{minipage}{0.45\linewidth}
\centering
	\begin{tikzpicture}[]
	\node[pattern-node, anchor=center] (vb1){VB};
	\node[pattern-node, anchor=center, left=3em of vb1, ] (in){TO};
	\path (vb1) edge[edge-style] node[above] {aux} (in);
	\end{tikzpicture}
	\caption{Correct infinitive}
	\label{fig:infinitive-final}
\end{minipage}
\end{figure}

\begin{figure}[!ht]
\begin{minipage}{0.45\linewidth}
\centering
	\begin{tikzpicture}[]
	\node[pattern-node, anchor=center] (vb1){VB};
	\node[pattern-node-negative, anchor=center, left=3em of vb1] (in){TO};
	\path (vb1) edge[edge-style] node[above] {aux} (in);
	\end{tikzpicture}
\caption{Present simple mislabelled as infinitive}
\label{fig:present-simple-init}
\end{minipage}
\quad
\begin{minipage}{0.45\linewidth}
\centering
	\begin{tikzpicture}[]
	\node[pattern-node, anchor=center] (vb1){VBP};
	\node[pattern-node-negative, anchor=center, left=3em of vb1] (in){TO};
	\path (vb1) edge[edge-style] node[above] {aux} (in);
	\end{tikzpicture}
	\caption{Correct present simple}
	\label{fig:present-simple-final}
\end{minipage}
\end{figure}
    
\subsection{Attributive verbs mislabelled as adjectives}
%7. correct_pos_of_predicate(dg):
%    """
%        if a node has a subject then it is a verb
%        eg: That    movie    upset/JJ    Ivy    .
%    """
In English, \textit{attributive verbs} often have the same lexical form as their corresponding adjectives. This is a reason for POS being mislabelled adjective(JJ) instead of verb (VB) leading to situations when an adjective (JJ) has an outgoing subject relation which means that its POS should actually be VB. The algorithm checks for such cases and corrects the JJ POS into VBP (non 3^{rd} person present simple).

\begin{figure}[!ht]
\begin{minipage}{0.45\linewidth}
\centering
	\begin{tikzpicture}[]
	\node[pattern-node, anchor=center] (vb1){VB};
	\node[pattern-node, anchor=center, left=5em of vb1] (in){X};
	\path (vb1) edge[edge-style] node[above] {aux} (in);
	\end{tikzpicture}
\caption{Mislabelled attributive verb}
\label{fig:attributive-verb-init}
\end{minipage}
\quad
\begin{minipage}{0.45\linewidth}
\centering
	\begin{tikzpicture}[]
	\node[pattern-node, anchor=center] (vb1){VBP};
	\node[pattern-node, anchor=center, left=5em of vb1] (in){X};
	\path (vb1) edge[edge-style] node[text centered, align=center] {nsubj,\\ nsubjpass} (in);
	\end{tikzpicture}
	\caption{Corrected attributive verb}
	\label{fig:attributive-verb-final}
\end{minipage}
\end{figure}

\subsection{Non-finite verbal modifiers with clausal complements}
%8. correct_misplaced_ccomp_on_partmod_node(dep_graph):
%    """
%        IF the partmod child node contains a ccomp link, then 
%        the ccomp link shall be placed on the parent  
%        
%        Eg: Tell the boy playing the piano that he is good. 
%    """
The early version of Stanford Dependencies \citep{Marneffe2008} proposes two relations for non-finite vernal modifiers \textit{partmod} for participial and \textit{infmod} for infinitival forms exemplified in \ref{ex:boy1}.% and \ref{ex:anything1}. 
Latter in \citep{Marneffe2014} both relations have been merged into the \textit{vmod}.

\begin{exe}
\ex\label{ex:boy1} Tell the boy playing the piano that he is good.
%\ex\label{ex:anything1} I don't have anything to say to you. 
\end{exe}

Clauses such as ``(that) he is good'' following immediately after the qualifier clause (``playing the piano'' in the example \ref{ex:boy1}) are problematic with respect to where shall they be attached: to the main clause or to the modifying one. This problem is similar to the prepositional phrase attachment problem. 

In this case, of course, attachment would depend on whether the verb accepts a clausal complement or not. In the example \ref{ex:boy1} the verb \textit{to play} does not take clausal complements then the clause ``that he is good'' is complementing ``tell the boy''. Stanford parser does not take into consideration such constraints and sometimes provides an incorrect attachment.

This type of error can be captured as the graph pattern in Figure \ref{fig:ccomp-vmod-init} which is transformed by the algorithm into the form represented in Figure \ref{fig:ccomp-vmod-final}
\begin{figure}[!ht]
\begin{minipage}{0.45\linewidth}
\centering
	\begin{tikzpicture}[]
	\node[pattern-node, anchor=center] (vb1){VB1};
	\node[pattern-node, anchor=center, below = 3em of vb1](nn){NN};
	\node[pattern-node, anchor=center, right=3.6em of nn](vb2){VB2};
	\node[pattern-node, anchor=center, right=3.5em of vb2](vb3){VB3};
	\path (vb1) edge[edge-style] node[right] {rel_{...}} (nn);
	\path (nn) edge[edge-style] node[above, text centered, align=center] {vmod} (vb2);
	\path (vb2) edge[edge-style] node[above, text centered, align=center] {ccomp} (vb3);
	\end{tikzpicture}
\caption{Clausal complement attached to the modifier clause}
\label{fig:ccomp-vmod-init}
\end{minipage}
\quad
\begin{minipage}{0.45\linewidth}
\centering
	\begin{tikzpicture}[]
		\node[pattern-node, anchor=center] (vb1){VB1};
		\node[pattern-node, anchor=center, below = 3em of vb1](nn){NN};
		\node[pattern-node, anchor=center, right=3.6em of nn](vb2){VB2};
		\node[pattern-node, anchor=center, right=2em of vb2](vb3){VB3};
		\path (vb1) edge[edge-style] node[right] {rel_{...}} (nn);
		\path (nn) edge[edge-style] node[above, text centered, align=center] {vmod} (vb2);
		\path (vb1) edge[edge-style] node[above, text centered, align=center] {ccomp} (vb3);
		\end{tikzpicture}
	\caption{Clausal complement attached to the main clause}
	\label{fig:ccomp-vmod-final}
\end{minipage}
\end{figure}

Syntactic structure is not enough to capture this error which originates in the semantic influences on the syntax. To grasp the them an extra constraint check is the possible lexico-semantic type of the verb. As only verbal and cognition process types can take clausal complements as phenomena then the verb in the higher clause VB_1 heeds to be capable of accepting clausal complement VB_2 before performing the reattachment. In other words, if VB_1 is capable of accepting two complements (i.e. di-transitive) then most likely the VB_2 is a complement, otherwise it is certainly not.

%\begin{figure}[!ht]
%\centering
%\begin{dependency}[dep-style]
%		\begin{deptext}[]
%	Tell \& the \& boy \& playing \& the \& piano \& that \& he \& is \& good \& . \\
%		\end{deptext}
%		\depedge{3}{4}{partmod}
%	\end{dependency}
%\caption{Participial verbal modifier}
%\label{fig:partmod1}
%\end{figure}
%\begin{figure}[!ht]
%\centering
%	\begin{dependency}[dep-style]
%		\begin{deptext}[]
%	I \& don \& 't \& have \& anything \& to \& say \& to \& you. \\
%		\end{deptext}
%		\depedge{5}{7}{infmod}
%	\end{dependency}
%\caption{infinitival verbal modifier}
%\label{fig:infmod1}
%\end{figure}


\subsection{Demonstratives with a qualifier}
%9. correct_DT_prep_NN(dep_graph):
%    """
%     if there is a DT(determiner) connected 
%     to a NN via preposition, then the NN is 
%     part of a PP that is a sibling of DT
%     
%     EG: ... and put that[DT] in the pan[NN]
%     NN shall de compl/adjunct of "put" and not a qualifier of DT
%    """
Demonstratives (\textit{this, that, these, those}) occurs as both determines and as pronouns. In English, when demonstratives are used as determiners, they function as Deictic element of a nominal group i.e. modifying the head of the nominal group. When used as pronouns, demonstratives never form phrases but occur as single words filling a clause element. Translated into dependency grammar demonstratives may have as parent either a noun (NN) or a verb (VB*). 

Examples below show uses of demonstratives in both cases. The word (thing/things)* enclosed between round brackets are not part of the sentence but are elipted.  

\begin{exe}
\ex\label{ex:demonstrative1} Bill moved those beyond the counter.
\ex\label{ex:demonstrative2} Put that in our plan.
\ex\label{ex:demonstrative3} Look at those (things)* beyond the counter.
\ex\label{ex:demonstrative4} What is that (thing)* next to the screen?
\ex\label{ex:demonstrative5} I thought those (things)* about him as well.
\ex\label{ex:demonstrative6} He felt that (thing)* as a part of him. 
\end{exe}
When demonstratives are followed by a prepositional phrase the question arises whether it shall be attached to the verb and take a clause role or it should be attached to the demonstrative as post-modifier. I shall note that demonstratives cannot take by themselves a post-modifier in either case as determiner or pronoun. 

However there are cases when apparently the post-modifier (prepositional phrase) pertains to the demonstrative like in the examples \ref{ex:demonstrative3} and \ref{ex:demonstrative4} and cases such as \ref{ex:demonstrative5} and \ref{ex:demonstrative6} when the post-modifier pertains to teh clause. 

In fact the only acceptable analysis for apparently a demonstrative with a Qualifier (i.e. post-modifier) can be analysed as noun phrases with the Thing missing (elipted) and the Deictic taking the role of the Head. The implied missing head is the generic noun ``thing(s)'' or any noun anaphorically binding the demonstrative.

The verb argument structure and syntactic constraints on the arguments described in the Transitivity classification of process types enable precise distinctions of such cases. However at this stage the algorithm does not employ this type of information. Therefore as a rule of thumb, the prepositional phrase following the demonstrative shall be attached to the verb in the case of non-projective\footnote{Projective verbs express cognitive and verbal processes like saying, thinking or imagining and often they verbs are di-transitive.} di-transitive verbs which are \textit{three role actions} and \textit{directional} processes.

In examples \ref{ex:demonstrative1} and \ref{ex:demonstrative2}, attaching the prepositional phrase to the demonstratives (depicted in Figure \ref{fig:demonstrative-init}) is incorrect. It should be attached to the verb (like in the figure \ref{fig:demonstrative-final}) because the prepositional phrase can function as Destination or Location in each case i.e take semantic roles. 

The algorithm detects cases of demonstratives that have attached a prepositional phrase. If the parent verb is a three role action or a directional process then the prepositional phrase is reattached to the verb.

\begin{figure}[!ht]
\begin{minipage}{0.45\linewidth}
\centering
	\begin{tikzpicture}[]
	\node[pattern-node, anchor=center] (vb1){VB\\(three role action, directional)};
	\node[pattern-node, anchor=center, below =2.5em of vb1,xshift=1em] (dt){DT \\ (demonstrative)};
	\node[pattern-node, anchor=center, below right =2.3em of dt,xshift=-2em] (in){IN};
	\node[pattern-node, anchor=center, right =1em of in] (nn){NN};
	\path (vb1) edge[edge-style] node[left] {dobj} (dt);
	\path (dt) edge[edge-style] node[above right] {prep\_xxx} (nn);
	\end{tikzpicture}
\caption{Prepositional phrase attached to the demonstrative determiner which is the head of a nominal group}
\label{fig:demonstrative-init}
\end{minipage}
\quad
\begin{minipage}{0.45\linewidth}
\centering
	\begin{tikzpicture}[]
	\node[pattern-node, anchor=center] (vb1){VB\\(three role action, directional)};
	\node[pattern-node, anchor=center, below =2.5em of vb1,xshift=1em] (dt){DT};
	\node[pattern-node, anchor=center, right =1em of dt] (in){IN};
	\node[pattern-node, anchor=center, right =1.5em of in] (nn){NN};
	\path (vb1) edge[edge-style] node[left] {dobj} (dt);
	\path (vb1) edge[edge-style] node[above right] {prep\_xxx} (nn);
	\end{tikzpicture}
	\caption{Prepositional Phrase attached to the verb with a demonstrative pronoun in between}
	\label{fig:demonstrative-final}
\end{minipage}
\end{figure}

Ideally, the algorithm should also change the POS of the demonstrative into pronoun but unfortunately Penn tag-set only contains personal and possesive pronouns. The demonstratives are always labelled as determiners so no POS change is made to the dependency graph but it is properly represented when converted into the constituency graph.

\subsection{Topicalized complements labelled as second subjects}   
%10. correct_two_subjects(dep_graph):
%    """
%        If there are two subjects then:
%        * give priority to the one closer to the verb 
%        * give priority to PR ovr NN OR
%        
%        current implementation takes only proximity to the verb into consideration 
%        
%        Ex: Any vitamins I could be lacking ?
%        det(vitamin-2, any-1)
%        nsubj(lack-6, vitamin-2)
%        nsubj(lack-6, I-3)
%        aux(lack-6, could-4)
%        aux(lack-6, be-5)
%        root(ROOT-0, lack-6)
%    """ 
In generative grammars the topicalization (or thematic fronting) of complements is described in \textit{Trace Theory} as \textit{WH/NP/PP-movement}. Examples \ref{ex:thematic-compl1}--\ref{ex:thematic-compl5} (from \citep[pp.~412-413]{Quirk1985}) present this phenomena. It is used in informal speech where it is quite common for an element to be fronted with a nuclear stress thus being informationally and thematically stressed. Alternatively this phenomena is used as a rhetorical style to point parallelism between two units and occurs in adjacent clauses like in examples \ref{ex:thematic-compl4}--\ref{ex:thematic-compl5}.

\begin{exe}
\ex\label{ex:thematic-compl1} Joe(,) his name is. 
\ex\label{ex:thematic-compl2} Relaxation(,) you call it. 
\ex\label{ex:thematic-compl3} Really good(,) cocktails they make at the hotel. 
\ex\label{ex:thematic-compl3.5} Any vitamins(,) I could be lacking? 
\ex\label{ex:thematic-compl4} His face(,) I'm not found of but his character I despise.
\ex\label{ex:thematic-compl5} Rich(,) I may be (but that does not mean I'm happy).
\end{exe}

These are difficult cases for Stanford parser (tested with versions up to 3.5.1). None of the above examples are parsed correctly. However, if the comma is present between topicalized complement and the subject, then it produces parses that are closest to the correct one where the topicalized complement is labelled as second subject but still not a complement. So having a comma present helps. 

\begin{figure}[!ht]
\begin{minipage}{0.45\linewidth}
\centering
	\begin{tikzpicture}[]
	\node[pattern-node, anchor=center] (vb1){VB\\(ditransitive)};
	\node[pattern-node, anchor=center, below left=4.3em of vb1,xshift=4em] (s1){NN};
	\node[pattern-node, anchor=center, left =1.3em of s1,xshift=0em] (s2){NN(,)};
	\node[pattern-node, anchor=center, below right =4.3em of vb1,xshift=-4em] (c1){NN/PP/VB};
	\path (vb1) edge[edge-style] node[right] {nsubj} (s1);
	\path (vb1) edge[edge-style] node[above left] {nsubj} (s2);
	\path (vb1) edge[edge-style] node[above right, xshift=0.5em, yshift=-1em] {dobj/iobj/ \\ prep/prepc} (c1);
	\end{tikzpicture}
\caption{Two consecutive nominal groups before the verb labelled as subjects}
\label{fig:topic-complement-init}
\end{minipage}
\quad
\begin{minipage}{0.45\linewidth}
\centering
	\begin{tikzpicture}[]
	\node[pattern-node, anchor=center] (vb1){VB\\(ditransitive)};
	\node[pattern-node, anchor=center, below left=4.3em of vb1,xshift=4em] (s1){NN};
	\node[pattern-node, anchor=center, left =1.3em of s1,xshift=0em] (s2){NN(,)};
	\node[pattern-node, anchor=center, below right =4.3em of vb1,xshift=-4em] (c1){NN/PP/VB};
	\path (vb1) edge[edge-style] node[right] {nsubj} (s1);
	\path (vb1) edge[edge-style] node[above left] {dobj} (s2);
	\path (vb1) edge[edge-style] node[above right, xshift=0.5em, yshift=-1em] {dobj/iobj/ \\ prep/prepc} (c1);
	\end{tikzpicture}
	\caption{Topicalized Direct Object -- moved to pre-subject position}
	\label{fig:topic-complement-final}
\end{minipage}
\end{figure}

The algorithm is looking for the cases of multiple subjects (represented in figure \ref{fig:topic-complement-init}) and gives priority to the one that is closest to the verb. The other one is relabelled as a complement (Figure \ref{fig:topic-complement-final}). The rule is generalized in the algorithm for multiple subjects even if so far only cases of two subjects have been observed.  

\subsection{Misinterpreted clausal complement of the auxiliary verb in interrogative clauses}
%3. correct_interrogative_ccomp_to_aux(result):
%    """ VB1-ccomp-VB2, VB2-[nsubj,csubj]-[PR,NN]; 
%        VB1<[PR,NN]<VB2; VB1 is modal-VB or aux-VB;
%        VB1 has no children 
%        then ccomp->aux
%    """
%TODO: explain

Sometimes the auxiliary verb in the interrogative clauses (examples \ref{ex:inter1} and \ref{ex:inter2}) is mistakenly used as a clause main verb. Instead of \textit{aux} relation from the main verb to the auxiliary there is a clausal complement relation from the auxiliary to the main verb.

\begin{exe}
	\ex\label{ex:inter1} Do you walk alone?.
	\ex\label{ex:inter2} Has Jane fed the cat?.
\end{exe}

The algorithm searched for the pattern depicted in Figure \ref{fig:aux-init} and transforms it into the form Figure \ref{fig:aux-final}. 

\begin{figure}[!ht]
\begin{minipage}{0.45\linewidth}
\centering
	\begin{tikzpicture}[]
	\node[pattern-node, anchor=center] (vb1){VB1};
	\node[pattern-node, anchor=center, below right=3em of vb1,xshift=-2em] (s){NN};
	\node[pattern-node, anchor=center, right=3em of s] (vb2){VB2};
	\path (vb2) edge[edge-style] node[below] {nsubj} (s);
	\path (vb1) edge[edge-style] node[above right] {ccomp} (vb2);
	\end{tikzpicture}
\caption{Mislabelled clausal complement}
\label{fig:aux-init}
\end{minipage}
\quad
\begin{minipage}{0.45\linewidth}
\centering
	\begin{tikzpicture}[]
	\node[pattern-node, anchor=center] (vb2){VB2};
	\node[pattern-node, anchor=center, below left=3em of vb2,xshift=0em] (s){NN};
	\node[pattern-node, anchor=center, left=1em of s] (vb1){VB1};
	\path (vb2) edge[edge-style] node[below right] {nsubj} (s);
	\path (vb2) edge[edge-style] node[above left] {aux} (vb1);
	\end{tikzpicture}
	\caption{Corrected clausal complement}
	\label{fig:aux-final}
\end{minipage}
\end{figure}

\section{Creation of systemic constituency graph from dependency graph}
\label{sec:creation-constituency-graph}
This section describes how the systemic constituency structure is generated from the dependency graph. Even if at first sight they appear isomorphic, DGs and CGs differ in their structure. The CG is created in two phases as presented in the Algorithm \ref{alg:creation-of-cg}. The first phase generates an incomplete CG through a top-down breadth-first traversal of a DG. The second phase complements the first one ensuring creation of all CG constituents through a bottom-up DG traversal.



\begin{algorithm}[H]
	\Input{ \dg (the dependency graph), \rt }
	\Output{ \cg (the constituency graph)}
	\Begin{
		create the bootstrap \cg by top-down traversal \;
		complete the \cg by bottom-up traversal \; 
	}
	\caption{Constituency graph creation}
	\label{alg:creation-of-cg}
\end{algorithm}

Before presenting the two stages of creation I will first reiterate over the difference in the dependency nature in the constituency and dependency graphs. Then I will also talk about the tight coupling of the two graphs and the rule tables used in traversal. 

%dependecy difference - justification for 1st phase skipping heads
\subsection{Dependency nature and implication on head creation}
\label{sec:dep-implications}
As explained in Section \ref{sec:dependecy-relations-sfl}, the nature of dependency relations is different in dependency graphs and in systemic functional constituency graphs. 

The DG uses a \textit{parent-daughter dependency} while in Constituency Graphs there is a \textit{sibling dependency}. This difference implies that, when mapped into CG, a DG node, stands for both a unit and that unit's head. In other words a DG node corresponds to two functions at different rank scales.For example the root verb in DG corresponds to the clause node and the lexical item which fills the Main Verb of the clause. 

In current approach, the top-down perspective considers the DG node as representing the upper most rank. The bottom-up perspective considers the DG nodes as representing the lower most rank. Thus the generation algorithm first traverses the graph in a top-down order and generating the units as appropriate and then bottom up in order to create their heads.

The result of the top-down phase is a constituency graph without head nodes. Therefore the bottom-up phase is performed by traversing th constituency graph and not on the dependency graph. The traversal task is to locally resolve which dependency nodes form the syntactic head. The local resolution is possible because of the tight coupling between the dependecy and constituency graphs established in the top-down phase. It is explained in the section below.

\subsection{Tight coupling of dependency and constituency graphs}
\label{sec:tight-coupling}
At the creation stage, the CG is tightly coupled with the original DG. This allows navigating easily from one graph to the other one via references stored within the nodes of each of them. I say that a graph node is \textit{aware} of it's ascription in another graph if it carries information to which nodes it is linked within the second graph.

%The DG nodes carry a stack of CG nodes resembling the rank scale order of units to which it belongs. 

Through a stack of CG nodes, the DG nodes are made aware of which CG nodes and in which order they subsume them (Figure \ref{fig:aware-dg}). On the other hand, the CG nodes are also made aware through a list of DG nodes, over which DG nodes do they span (Figure \ref{fig:aware-mcg}). This way the positioning information is available bidirectionally about CG an DG structures.

\begin{figure}[!ht]
	\begin{minipage}{0.45\linewidth}
		\centering
		\begin{tikzpicture}[]
		\node[pattern-node, ] (vb1){\textit{smiled}\\(clause1,\\ mainVerb1)\\};
		\node[pattern-node, left=0.2em of vb1] (s){\textit{girl}\\(clause1,\\subject1,\\thing1)};
		\node[pattern-node, left=0.2em of s] (det){\textit{The}\\(clause1,\\subject1,\\determiner1)};
		\path (vb1) edge[edge-style, in=80,out=100] node[above] {nsubj} (s);
		\path (s) edge[edge-style,in=80,out=100] node[above] {det} (det);
		\end{tikzpicture}
		\caption{Constituency aware DG nodes}
		\label{fig:aware-dg}
	\end{minipage}
	\quad
	\begin{minipage}{0.45\linewidth}
		\centering
		\begin{tikzpicture}[]
		\node[pattern-node, anchor=center, text width=18em] (cl){clause1\\(\textit{The, girl, smiled})}
		child { node[pattern-node, text width=11em, below =4em of cl.west, anchor=west] (s){subject1\\(\textit{The, girl})} 
			child { node[pattern-node,below =4em of s.west, anchor=west](d) {determiner1\\(\textit{The})} }
			child { node[pattern-node, text width=4em, below =4em of s.east, anchor=east](h) {thing1\\(\textit{girl})} } }
		child {node[pattern-node,text width=5em, below =4em of cl.east, anchor=east] {mainVerb1\\(\textit{smiled})}};
		\end{tikzpicture}
		\caption{Dependency aware CG nodes}
		\label{fig:aware-mcg}
	\end{minipage}
\end{figure}

The Figure \ref{fig:aware-dg} depicts a dependency graph. Each node has a stack of ids corresponding to constituents in the CG (Figure \ref{fig:aware-mcg}). Conversely, in Figure \ref{fig:aware-mcg}, depicts a constituency graph where the constituent nodes carry a set of tokens corresponding to DG nodes. This way the DG in Figure \ref{fig:aware-dg} and the CG in \ref{fig:aware-mcg} are aware of each other. 

The node awareness has two interesting properties worth exploring. First, the DG nodes receive a vertical constituency strip. Each strip is a direct path from the root to the bottom of the constituency graph where the word of the DG is found. These strips are the very same ones explored in the parsing method explored by \citet{Day2007}.    
Second, the CG nodes receive a horizontal span over DG nodes enabling exploration of elements linear order. These two properties could eventually be explored in future work to inform or verify the correctness of the constituency graph.

% graphs are created according to rule tables
\subsection{Mapping Rule Tables}
\label{sec:mapping-rule-table}

Constituency Graphs are created through a top-down breadth-first walk of dependency graph. During the traversal each visited edge triggers execution of a \textit{creational operation} on the growing CG on the side. To know what operation to execute a rule table is used where the edge type, head and tail nodes are mapped to a creational operation and eventually a parameter (specifying the element type if a constituent is to be created).

A simplified example of the rule table is presented in Table \ref{tab:rule-table}. It can be regarded as an attribute value matrix or a Python dictionary where the left column titled: key, contains a unique \textit{dependency graph context} serving as a rule trigger; while the left side column named: value, contains the operation to be executed within the given context. 

Current implementation uses three operation types: (a) \textit{creating a new constituent} under a given  one (b) \textit{creating a new sibling} to the given one (c) \textit{extend} a constituent with more dependency nodes. 

The parameter is used only for the operations (a) and (b) and specifies which element the new constituent is filling as described in Section \ref{sec:sydney-theory-of-grammar} and \ref{sec:cardiff-theory-grammar}. Most of the time there is only one element provided but in the case of prepositional phrases and clauses it is impossible to specify purely on syntactic basis the exact functional role and thus multiple options are provided (Adjunct or Complement) and then in lather parsing phases, which account for the verb semantics, these options ideally are reduced to one.  

%TODO refer to teh full rule tables in Appendix
%\begin{table}[!ht]
%\centering
%\begin{tabular}{|l|l|l|l|}
%\hline
%  & \textbf{Key} & \textbf{Operation}       & \textbf{Parameter}  \\ \hline
%1 & nsubj        & new\_constituent         & Subject             \\ \hline
%2 & csubj        & new\_clause\_constituent & Subject             \\ \hline
%3 & prepc        & new\_clause\_constituent & Complement, Adjunct \\ \hline
%4 & VB-prep-NN   & new\_constituent         & Complement, Adjunct \\ \hline
%5 & NN-prep-NN   & new\_constituent         & Qualifier           \\ \hline
%6 & VB-advmod-WR & new\_constituent         & Complement          \\ \hline
%7 & VB-advmod-RB & new\_constituent         & Adjunct             \\ \hline
%8 & mwe          & extend\_current          &                     \\ \hline
%9 & nn           & extend\_current          &                     \\ \hline
%\end{tabular}
%\caption{Rule table example mapping (specific or generic) dependency context to constructive operation}
%\label{tab:rule-table}
%\end{table}
%

% Please add the following required packages to your document preamble:
% \usepackage{multirow}
\begin{table}[!ht]
	\centering
	\begin{tabular}{|l|c|l|l|}
		\hline
		\multirow{2}{*}{} & \multirow{2}{*}{\textbf{Key}} & \multicolumn{2}{c|}{\textbf{Value}}                                                                 \\ \cline{3-4} 
		&                               & \multicolumn{1}{c|}{\textit{\textbf{Operation}}} & \multicolumn{1}{c|}{\textit{\textbf{Parameter}}} \\ \hline
		1                 & nsubj                         & new\_constituent                                 & Subject                                          \\ \hline
		2                 & csubj                         & new\_clause\_constituent                         & Subject                                          \\ \hline
		3                 & prepc                         & new\_clause\_constituent                         & Complement, Adjunct                              \\ \hline
		4                 & VB-prep-NN                    & new\_constituent                                 & Complement, Adjunct                              \\ \hline
		5                 & NN-prep-NN                    & new\_constituent                                 & Qualifier                                        \\ \hline
		6                 & VB-advmod-WR                  & new\_constituent                                 & Complement                                       \\ \hline
		7                 & VB-advmod-RB                  & new\_constituent                                 & Adjunct                                          \\ \hline
		8                 & mwe                           & extend\_current                                  &                                                  \\ \hline
		9                 & nn                            & extend\_current                                  &                                                  \\ \hline
	\end{tabular}
	\caption{Rule table example mapping (specific or generic) dependency context to constructive operation}
	\label{tab:rule-table}
\end{table}

There are two types of keys in the rule table: the \textit{generic} ones where the key consist of the (non-extended) dependency relation and the \textit{specific} ones surrounded by the POSes of head and tail edge nodes taking the form \mbox{\textit{Tail--relation--Head}}. For example \textit{nsubj} relation (on row 1) always leads to creation of a Subject nominal constituent regardless if it is headed by a noun, pronoun or adjective. Since all individual cases lead to the same outcome it suffices to map the dependency relation to the creation of a new constituent with Subject role ignoring the POS context of head and tail nodes. The same holds for \textit{prepc} relation (on the table row 3) as it always leads to creation of subordinate clause constituent with Complement or Adjunct roles. So the generic relations can be viewed as equal to the form \mbox{\textit{Any--relation--Any}} only that the nodes are omitted due to redundancy.

In the case of \textit{prep} relation (on the rows 4 and 5) the story is different. Its interpretation is highly dependent on its context given by the parent/tail and child/head nodes. If it is connecting a verb and a noun then the constituent prepositional phrase takes the role of either Complement or Adjunct in the clause. But if the prep relation is from a noun to another noun, then it is a prepositional phrase with Qualifier function in the nominal group.

Some dependency relation are not mapped to operation of creating a new but rather extend the existing constituent with all nodes succeeding current one in the DG. These operation is used for two reasons: either (a) the constituent truly consists of more than one word, for example the cases of multi word expressions (e.g. ice-cream) marked via \textit{mwe} relation (table-row 8) or (b) the relation (with or without it's POS context) is insufficiently informative for instantiating a constituent node and is postponed for the second phase of the CG creation.

Note that the contextualised relations are ``slightly'' generalised by reducing POS to first two letters which can be up to four letters long. For example nouns generically are marked as NN but they may be further specified as NNP, NNPS and NNS or verbs (VB) may be marked as VBD, VBG, VBN, VBP, and VBZ depending on their form. 

Next I explain the top-down traversal phase which is the core essence of the constituency graph creation. 

\subsection{Top down traversal phase}
\label{sec:first-phase}

The goal of this first phase is to bootstrap a partial constituency graph starting from a given dependency graph and a table with mapping rules. The CG is created as ap parallel structure through the process of breadth-first traversal on DG edges as described in Algorithm \ref{alg:phase-one}.

% phase one
\begin{algorithm}[H]
	\Input{ \dg (the dependency graph), \rt }
	\Output{ \cg (the constituency graph)}
	\Begin{
		create the \cg with a root node\;
		make the \cg root node aware of the \dg root node\;
		\For{ \edge \KwTo list of \dg edges in BFS order}
		{
			\rrule $\leftarrow$ find the suitable rule for the current edge in the \rt \label{line:choose-oper} \;
			\operation $\leftarrow$ get operation from the rule \;
			\elementType $\leftarrow$ if any get the parameter from the rule \label{line:param} \;
			\tcp{as \cg and \dg nodes are aware of each other we can navigate between them}
			\stack $\leftarrow$ the constituency stack from the tail node of the current \dg edge \;
			\cgPointer $\leftarrow$ the top node from the \stack \;
			\Children $\leftarrow$ all child nodes for the current dg edge \label{line:children}\;
			\tcp{ constructing or extending the \cg with a new node}
			execute the \operation on \cg given \elementType, \cgPointer and \Children \label{line:execute-oper}\;
		}
		\Return{\cg} \;
	}
	\caption{Top-down CG creation}
	\label{alg:phase-one}
\end{algorithm}

First the CG is instantiated and an empty root node is created within the CG. Also, the root node is made aware of the root node in DG via the mechanism described in Section \ref{sec:tight-coupling}.

Then the DG is traversed on its the edges in BFS order starting from the root node. As each DG edge is visited an operation is chosen based on the edge type and nodes POS along with computation of an additional set of parameters Lines \ref{line:param} - \ref{line:children}; after which the creative operation is executed with the established parameters: dependency successors (\Children), a constituent node parenting the newly created one (\cgPointer), \elementType which is chosen from the rule-table together with the \operation. Note that the head nodes are not created in current but the next phase. 

The Line \ref{line:choose-oper} of the algorithm is responsible for looking up and selecting the \operation from the \rt as described in Algorithm \ref{alg:look-up}. It is based on two lookups based on the rule indexes: one contextualised to the edge relation and its nodes and another one generic based on the edge relation alone.

The \rt is conceived as a Python dictionary with string keys and and two-tuple containing the \operation and the \elementType parameter. If the key is found (either specific or generic) in the \rt then the operation and parameter are returned otherwise None is returned. 

\begin{algorithm}[H]
\Input{ \rt, \edge}
\Output{ \rrule}
\Begin{
%	\tcp{simple key is the first segment of DG relation}
	\simpleKey $\leftarrow$ the simplified dependency relation of the \edge   \;
	\headPos $\leftarrow$ the POS of the \edge head node \;
	\tailPos $\leftarrow$ the POS of the \edge tail node \;
%	\tcp{specific/contextualised key is the simple key surrounded by the edge 
%					nodes part of speech trimmed to their first two characters}
	\ctxKey $\leftarrow$ \tailPos + \simpleKey + \headPos \;
	\If{\ctxKey index \KwTo \rt}
		{
			\rrule $\leftarrow$ rule indexed with \ctxKey from \rt \;
		}
	\ElseIf {\simpleKey index \KwTo \rt}
		{
			\rrule $\leftarrow$ rule indexed with \simpleKey from \rt \;
		}
	\Else
		{
			\rrule $\leftarrow$ None \;
		}
	\Return{\rrule}
	}
	\caption{Operation selection in the mapping rule table based on the edge type}
	\label{alg:look-up}
\end{algorithm}

In Python function are first class objects allowing objects to be called (executed) if they are of callable typed. This duality allows storing the functions directly in the \rt and then, upon lookup they are returned as objects but because they are also callable these objects are executed with the expected set of parameters based on their function aspect. The possible operation have been already explained in Section \ref{sec:mapping-rule-table}: \textit{extend current} and \textit{new (sibling) constituent}. Next I present the pseudo-code for each of them. Note that these operations do not return anything because their effect is on the input \cg and \dg. 

% extend oper
\paragraph{Extend constituent.} Algorithm \ref{alg:extend-current} outlines the functionality for extending current \cgPointer. It does two main things. It increases the span of an already existing CG node over more DG nodes concomitantly making them aware of each other. 

\begin{algorithm}[H]
\Input { \cgPointer, \Children, \elementType, \edge, \dg, \cg}
\Begin{
	\tcp{handling special relations \textit{prep} and \textit{conj}}
	\If{$prep \vee conj$ \KwTo \edge relation }
		{
			\fn $\leftarrow$ find in \dg the free nodes refereed in the \edge relation \label{line:free-node} \;
			create new node with \textit{marker} function under the \cgPointer with \fn as children \label{line:new-marker} \;
			\Children $\leftarrow$ \Children \& \fn \;
		}
		\tcp{making the \Children and \cgPointer aware of each other}
		\For {\node \KwTo \Children}
			{
				\stack $\leftarrow$ the constituency stack of the \node \;
				push the \cgPointer to the \stack \;
				\sspan $\leftarrow$ constituent span of the \cgPointer \;
				extend the \sspan with current \node \;
			}
	}
	\caption{Extend a constituent with DG nodes}
	\label{alg:extend-current}
\end{algorithm}

If, however, the \edge relation is a conjunction or a preposition then the children list is extended with the free nodes that stand for the preposition or conjunction in place and eventually neighbouring punctuation marks (line \ref{line:free-node}). 

This exceptional treatment is due to the fact that \textit{prep} and \textit{conj} relations are always specialised by the preposition or conjunction in place. For details on this aspect of Stanford Dependency Grammar please refer to Section \ref{sec:collapsed-cc-output}.

\begin{figure}
\centering
\begin{minipage}[b]{0.45\textwidth}
\centering
	\begin{dependency}[dep-style]
		\begin{deptext}[]
	He \& went \& on \& vacantion \& . \\
		\end{deptext}
		\deproot{2}{root}
		\depedge{2}{1}{nsubj}
		\depedge{2}{4}{prep\_on}
	\end{dependency}
\end{minipage}
\begin{minipage}[b]{0.45\textwidth}
	\begin{dependency}[dep-style]
		\begin{deptext}[]
	He \& went \& on \& vacation \& on \& a \& camel \& . \\
		\end{deptext}
		\deproot{2}{root}
		\depedge{2}{1}{nsubj}
		\depedge{2}{4}{prep\_on}
		\depedge{2}{7}{prep\_on}
		\depedge{7}{6}{det}
	\end{dependency}
\end{minipage}
\caption{Challenging free nodes}
\label{fig:challenging-free-nodes}
\end{figure}

Figure \ref{fig:challenging-free-nodes} exemplifies easy (on the left) and more difficult cases (on the right) of free node occurrence. The challenge in the second figure comes from the fact that there may be two suitable free nodes for the edge \textit{went-prep\_on-camel}. Another challenge type in resolving free nodes is when the preposition is a multi word construction. 

Once all the free DG nodes are found, a new \cg node is created with \textit{Marker} \elementType spanning over them (line \ref{line:new-marker}). Then the free nodes are included into the list of \Children and all together are made aware of the current \cgPointer and vice versa. 

% Create new
\paragraph{Create new constituent.} The Algorithm \ref{alg:create-new} is a function that extends CG with a newly created constituent object (line \ref{line:add-to-graph}). The new constituent is created as a child of a pointed CG node with the \elementType extracted from the \rt together with the \operation. Once the \cg node is created, it is extended with the \Children nodes as described in Algorithm \ref{alg:extend-current} above.

Note that only the functional element is assigned to the freshly created constituent. It's class is added in the second phase of the creation algorithm. This is due to the fact that a function can be filled by units of several classes. The bottom up traversal provides a holistic view on the constituency of each unit giving the possibility to assign a class accordingly. For details see the Chapter \ref{ch:sfg}.

\begin{algorithm}[H]
\Input { \cgPointer, \Children, \elementType, \edge, \dg, \cg}
\Begin{
	\node $\leftarrow$ new Constituent \;
	\node type $\leftarrow$ \elementType \;
	%\node parent $\leftarrow$ \cgPointer \;
	add to \cg the edge (\cgPointer, \node) \label{line:add-to-graph} \;
	\tcp{invoking the Algorithm \ref{alg:extend-current}}
	extend \cgPointer with \Children of \edge \;
	}
	\caption{Creating new child constituent}
	\label{alg:create-new}
\end{algorithm}

A variation of the \textit{create new} is \textit{create sibling}  outlined in the Algorithm \ref{alg:create-new-sibling}. It sets the newly created constituent as a sibling of the current one and not as a child. This will make the new constituent a child of current \cgPointer's parent.

\begin{algorithm}[H]
	\Input { \cgPointer, \Children, \elementType, \edge, \dg, \cg}
	\Begin{
		\cgPointer $\leftarrow$ get the parent of \cgPointer \;
		\tcp{invoking the Algorithm \ref{alg:create-new}}
		create new constituent to an updated \cgPointer \;
	}
	\caption{Creating new sibling constituent}
	\label{alg:create-new-sibling}
\end{algorithm}
% create sibling 

\subsection{Bottom up traversal phase}
Chapter \ref{ch:sfg} explains that each constituent must specify the unit class and the element it is filling within parent unit. The first phase of the algorithm achieves creating most of the constituents and assigns each unit functional elements derived from the dependency graph. 

The constituency graph misses, however, the unit classes and the syntactic head nodes. The second phase complements the first one by fulfilling two goals: (a) creation of constituents skipped in the first phase and (b) class assignment to the constituent units.
\begin{figure}[H]
	\centering
	\begin{dependency}[dep-style]
		\begin{deptext}[]
			He \& could \& have \& tried \& to \& unlock \& the \& door \&. \\
		\end{deptext}
		\deproot{4}{root}
		\depedge{4}{1}{nsubj}
		\depedge{4}{2}{aux}
		\depedge{4}{3}{aux}
		\depedge{6}{5}{aux}
		\depedge{4}{6}{xcomp}
		\depedge{8}{7}{det}
		\depedge{6}{8}{dobj}
	\end{dependency}
	\caption{The dependency graph before the first phase}
	\label{fig:dep-example1}
\end{figure}

The Figure \ref{fig:post-1st-phase-mcg} depicts and example CG generated in the 1^{st} phase with dotted lines representing places of the missing constituents. 

\begin{figure}[H]
	\centering
	\begin{tikzpicture}[]
	\node[pattern-node, anchor=center, text width=27em] (cl){Clause\\(\textit{He could have tried to unlock the door.})}
		child { node[pattern-node, text width=5em, below =4em of cl.west, anchor = west] (s){Subject\\(\textit{He})} }
	  	child {node[pattern-node,text width = 14em, below =4em of cl.east, anchor = east](c1) {Complement\\(\textit{to unlock the door.})} 
			  	child { node[pattern-node,text width = 8em, below =4em of c1.east, anchor = east](c2) {Complement\\(\textit{the door.})} 
			  		child { node[pattern-node,below =4em of c2.west, anchor = west](d) {Deictic\\(\textit{the})} }
	  			 }
	  		};
	\coordinate[right=1em of s.east] (cor1);
	\coordinate[left=1em of c1.west] (cor2);
	\coordinate[left=5em of c2.west] (cor3);
	\coordinate[left=1em of c2.west] (cor4);
	\coordinate[right=1em of d.east] (cor5);
	\coordinate[right=4em of d.east] (cor6);
	
	\draw[dashed] (cor1) -- (cor2);
	\draw[dashed] (cor3) -- (cor4);
	\draw[dashed] (cor5) -- (cor6);
	
	\end{tikzpicture}
	\caption{Constituency graph after the top down traversal missing the head nodes}
	\label{fig:post-1st-phase-mcg}
\end{figure}

The missing constituents are the syntactic heads for all units. The clause, besides the Main Verb, also misses the Finite, Auxiliary elements. Determining these functions strongly depends on the place within a unit and syntagmatic order in which units occur. As the first phase is performed as graph traversal, the order dimension is not available so they have to be created in the second phase. 

The class membership of constituent units is decided based on three informations available within each constituent: (a) part o speech of the head dependency node (b) element type of the constituent and (c) presence or absence of child constituents and their element types. The corresponding constraints are listed in Table \ref{tab:class-constitions}. The first column carries the unit class (to be assigned), the second and third columns enumerate part of speech and element types that constituents might fill. The Second and third column enumerations are exclusive disjunction sets (S_{XOR}) because only one may be selected at a time while the list in last column is an open disjunction (S_{OR}) because any of child elements may be present. The last column is an enumeration of what child constituents might the current one have. The \textit{n/a} means that information is unavailable. 

\begin{table}[H]
\centering
\begin{tabulary}{\textwidth}{|L|L|L|L|}
\hline
\textit{Class} & \textit{POS of the Head DG Node (XOR)} & \textit{Element Type (XOR)} & \textit{Child Constituents (OR)} \\ \hline
Clause              & VB* & Subject, Complement, Qualifier, n/a & Subject, Complement, Adjunct, n/a \\ \hline
Prepositional Group & CD, NN*, PR*,WP*, DT, WD* & Complement, Qualifier, Adjunct  & Marker, n/a \\ \hline
Nominal Group       & CD, NN*, PR*,WP*, DT, WD*, JJ, JJS & Subject, Complement & Deictic, Numeral, Epithet, Classifier, Qualifier, n/a \\ \hline
Adjectival Group    & JJ* & Complement, Epithet, Classifier & Modifier, n/a \\ \hline
Adverbial Group     & RB*, WRB & Adjunct & Modifier, n/a \\ \hline
\end{tabulary}
\caption{Constraints for unit class assignment}
\label{tab:class-constitions}
\end{table}

The Algorithm \ref{alg:phase-two} traverses the CG bottom-up (not the DG) with \textit{post-order depth-first} order (line \ref{line:iterate-leafs}) during which every visited constituent node is assigned a unit class and missing child constituents are created.

Because the CG and DG are tight coupled which means that each CG node spans over a set of DG nodes then traversing CG is equal to traversing groups of DG nodes at each step. This creates a focused mini context suitable for resolving the unit class and missing elements. 

When assigning unit class the following informations are considered: (a) part of speech of the head DG node that triggered node creation in the first phase (\headPos), (b) the assigned element type (\elementType) and (c) element type of each direct child (\Children of \node). Lines \ref{line:start-class-assignment} to \ref{line:end-class-assignment} assign classes according to conditions stated in Table \ref{tab:class-constitions}. 

Clause classes are assigned to units started by a non-modal verb. This rule corresponds to the one main verb per clause principle discussed in Section \ref{sec:verbal-grpoup-and-clause-division}. This approach however does not take into consideration elliptic clauses and they need additional resolution. This short coming should be considered in the future by an additional \textit{ellipsis resolution mechanism}, similar to the one for \textit{null elements} described in Chapter \ref{ch:gbt}.

The second part of algorithm creates head nodes for every non leaf constituent and in case the node is a clause then it also creates the clause elements such as: Finite, Auxiliary, Main Verb, Negator and Extension.

\begin{algorithm}[!ht]
\Input {\cg, \dg}
\Begin{
	\For {\label{line:iterate-leafs}\node \KwTo list of \cg nodes in DFS post-order}
		{
			\headPos $\leftarrow$ POS of the \dg entry node of \node \;
			\elementType $\leftarrow$ currently assigned element of \node \;
			\Children $\leftarrow$ get assigned elements to the children of \node \;
			\tcp{assigning classes}
			\If {\label{line:start-class-assignment}\headPos \KwTo main verb POSs}
				{
					assign \node \textit{Clause} class \;
				}
			\ElseIf {\headPos \KwTo nominal POSs $\wedge$ \elementType \KwTo prepositional element types $\wedge$ Marker \KwTo \Children}
				{
					assign \node \textit{Prepositional Group} class \;
				}
			\ElseIf {\headPos \KwTo nominal POSs $\wedge$ \elementType \KwTo nominal elemement types}
				{
					assign \node \textit{Nominal Group} class \;
				}
			\ElseIf {\headPos \KwTo adverbial POSs $\wedge$ \elementType \KwTo adverbial element type}
				{
					assign \node \textit{Adverbial Group} class \;
				}
			\ElseIf {\headPos \KwTo adjectival POSs $\wedge$ \elementType \KwTo adjectival element type}
				{
					assign \node \textit{Adjectival Group} class \label{line:end-class-assignment} \;
				}
			\tcp{creating the rest of the units}
			\If{\label{line:is-mcg-leaf} \node is not a leaf}
				{
					\eIf{\node is a Clause}
						{	
							\label{line:create-clause-nodes} create the Clause elements for current \node \;
						}
						{
							\label{line:create-head-node} create the head for the current \node \; 
						}
				}
		}
	}
\caption{Creating the head units and assigning classes}
\label{alg:phase-two}
\end{algorithm}

After the second stage, all the sentence token must be covered by CG nodes. Moreover the CG nodes build up to a constituency graph that at this stage is always a tree. Provided the class and element type the nodes are ready to be enriched with choices from systemic networks described in the next section. 

\section{Feature enrichment process}
\label{sec:enrichment-stage}
In this stage the CG nodes are assigned features from system networks. This is achieved by visiting each CG node in a bottom-up order and based on the node class and/or element function (also refereed to as triggers) the relevant system networks are being activated and executed. The relevancy criteria is established by system's precondition set. Each network has one or a set of selector functions associated to it and when the network is activated then the selector function is executed returning the systemic choices based on the visited node context in the CG graph and features (if any already assigned).

Table \ref{tab:systemic-selection-activations} associates a list of triggers to a list of systemic networks. Table \ref{tab:system-to-function-mapping} associates systems with the choice making functions. Most of the selection criteria have been first implemented as hard-coded Python functions and latter transformed into the Graph patterns with update operations (see Section \ref{sec:pattern-based-operations} describing pattern based operations).

Next I describe the enrichment algorithm and then treat some of the selection functions for some of the system networks. 

\begin{algorithm}[!ht]
\Input {\cg, \dg}
\Begin {
	\For{\node \KwTo list of  \cg nodes in DFS postorder}
	{
		\For{\network \KwTo activated networks for the \node} 
		{
			\function $\leftarrow$ the selector function associated to the \network \;
			\elementType $\leftarrow$ if any get the \function parameter \;
			\label{line:selector-function}\choices $\leftarrow$  execution result of \function for \node, \cg and \elementType \;
			\label{line:enrich-function}add the \choices to the \node \;
		}
	}
}
\caption{Enriching CG with systemic features implemented as custom methods}
\label{alg:enrich-mcg}
\end{algorithm}

The Algorithm \ref{alg:enrich-mcg} shows how to enrich CG nodes with systemic choices using hard-coded selector functions (in the future, these hard coded selector functions will be replaced by verifying realization statements). The outer loop is a bottom-up iteration over the CG nodes in depth first (DFS) post-order. The inner loop iterates over the networks activated by the focus \node. Then a lookup in the Table \ref{tab:system-to-function-mapping} returns the function for choosing features from the system network and eventually a parameter (if the function requires it). This technique, involving a mapping table from SN to functions, is similar to the one in CG creation phase (Algorithm \ref{alg:phase-one}). The line \ref{line:selector-function} executes the selection function returning a set of choices. Then the last line assigns the choices to the \node in focus.

\begin{table}[H]
\centering
\begin{tabulary}{\textwidth}{|L|L|}
\hline
\textit{Key} & \textit{System Activation Order}                                                \\ \hline
clause       & POLARITY,VOICE, AGENCY, MOOD TYPE, INDICATIVE TYPE, DEICTICITY, TENSE, MODALITY \\ \hline
nominal      & PERSON, ANIMACY, GENDER, NUMBER                                                 \\ \hline
deictic      & DETERMINATION                                                   				\\ 
\hline
pre-deictic  & DETERMINATION                                                                   \\ \hline
thing        & PERSON, ANIMACY, GENDER, NUMBER                                                 \\ \hline
possessor    & PERSON, ANIMACY, GENDER, NUMBER                                                 \\ \hline
\end{tabulary}
\caption{System activation table depending on unit class or element type}
\label{tab:systemic-selection-activations}
\end{table}

Currently implemented networks are outlined in Table \ref{tab:systemic-selection-activations} as associations to the node class or function. The left column represents activation triggers and the right column represents an ordered list of systems.

\begin{table}[H]
\centering
\begin{tabulary}{\textwidth}{|L|L|L|}
\hline
\textit{System Name} & \textit{Python Function}   & \textit{Additional Parameter} \\ \hline
POLARITY             & check\_polarity     &                               \\ \hline
VOICE                & check\_voice        &                               \\ \hline
AGENCY               & check\_agency       &                               \\ \hline
MOOD TYPE            & mood\_type          &                               \\ \hline
DEICTICITY           & check\_deicticity   &                               \\ \hline
TENSE                & check\_tense        &                               \\ \hline
MODALITY             & check\_modality     &                               \\ \hline
DETERMINATION        & dictionnary\_lookup & determination\_dict           \\ \hline
PERSON               & dictionnary\_lookup & person\_dict                  \\ \hline
ANIMACY              & dictionnary\_lookup & animacy\_dict                 \\ \hline
GENDER               & dictionnary\_lookup & gender\_dict                  \\ \hline
NUMBER               & check\_number       &                               \\ \hline
MOOD ADJUNCT TYPE    & dictionary\_lookup & mood\_adjunct\_dict           \\ \hline
\end{tabulary}
\caption{Mapping systemic networks to selection functions}
\label{tab:system-to-function-mapping}
\end{table}

%Once the system is activated its selector function (Table \ref{tab:system-to-function-mapping}) returns the set of choices.

\begin{algorithm}[!ht]
	\Input {\cg, \dg}
	\Begin {
		\For{\node \KwTo list of  \cg nodes in DFS postorder}
		{
			\For{\network \KwTo activated networks for the \node} 
			{
				\function $\leftarrow$ the selector function associated to the \network \;
				\elementType $\leftarrow$ if any get the \function parameter \;
				\label{line:selector-function-dl}\choices $\leftarrow$  execution result of \function for \node, \cg and \elementType \;
				\label{line:enrich-function-dl}add the \choices to the \node \;
			}
		}
	}
	\caption{Dictionary lookup selector function}
	\label{alg:dictionary-loockup}
\end{algorithm}

The \textit{dictionary\_lookup} function is a type of \textit{naive backwards induction} (Algorithm \ref{alg:backward-induction-naive}). It checks whether the word(s) of the \textit{mcg\_node} are in a dictionary with preselected features. It is the only function that requires a parameter which is the lookup dictionary. An example of dictionary is presented in the Table \ref{tab:lookup-dict-example}

\begin{table}[!ht]
	\centering
	\begin{tabular}{|l|l|}
		\hline
		\textit{Lexical item} & \textit{Feature}               \\ \hline
		a                     & partial-non-selective-singular \\ \hline
		all                   & positive-plural                \\ \hline
		either                & partial-singular               \\ \hline
		that                  & non-plural                     \\ \hline
	\end{tabular}
	\caption{Dictionary example for the DEICTICITY system network}
	\label{tab:lookup-dict-example}
\end{table}

\section{Discussion}
\todo[inline]{* replace hard coded selector functions with constraint checking }

\todo[inline]{ assign feature with a certain probability and allow assignments of mutually exclusive features; this would require the next step of more general constraint checking employing many more other sources such as semantic, phonetics, situation etc.  }

\todo[inline]{use of logical notations (OR, XOR, AND sets)} 

