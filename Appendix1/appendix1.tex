\chapter{SFL Syntactic Overview}
\label{ch:syntax-overview}

\section{Cardiff Syntax}
\noindent\textbf{Elements found in all groups:} Linker (\&), Inferer (I), Starter (st), Ender (e)

\noindent\textbf{Units:} Sentence ($\Sigma$), Clause (Cl), Nominal Group (ngp), Prepositional Group (pgp), Quality Group (qlgp), Quantity Group (qtgp), Genitive Cluster (gencl)

\subsection{Clause}

\textbf{Relative Order of Elements in the Unit Structure:}\\ \noindent
\&~|B~|L~|F~|A~|C~|O~|S~|O~|N~|A~|I~|X~|M~|Mex~|C~|A~|V~|E

\noindent\textbf{Clause May fill:}
$\Sigma$ (85\%), C (7\%), A (4\%), Q (2\%), f (0.5\%), s, qtf, S, m, cv, po

\noindent\textbf{Elements of the Clause:}
Adjunct (A), Binder (B), Complement (C), Formulaic Element (F), Infinitive Element (I), Let Element (L),Main Verb (M), Main Verb Extension (Mex), Negator (N), Operator (O), Subject (S), Vocative (V), Auxiliary Verb (A), X extension (Xex), Linker (\&), Starter (St), Ender(E)

\subsection{Nominal Group}
\textbf{Possible Relative Order of Elements in the Unit Structure:} \\ \noindent
\&~|rd~|v~|pd~|v~|qd~|v~|sd~|v~|od~|v~|td~|v~|dd~|m~|h~|q~|e

\noindent\textbf{Filling probabilities of the ngp:} S (45\%), C (32\%), cv (15\%), A (3\%), m (2\%), Mex, V, rd, pd, fd, qd, td, q, dt, po

\noindent\textbf{Elements of the ngp:} Representational determiner (rd), Selector (v), Partitive Determiner (pd), Fractionative Determiner (fd), Quantifying Determiner (qd), Superlative Determiner (sd), Ordinative Determiner (od), Qualifier-Introducing Determiner (qid), Typic Determiner (td), Deictic Determiner (dd), Modifier (m), Head (h), Qualifier (q)

\subsection{Prepositional Group}
\textbf{Possible Relative Order of Elements in the Unit Structure:} \\ \noindent
\&~|pt~|p~|cv~|p~|e

\noindent\textbf{Filling Probabilities of the pgp:} C (55\%), a (30\%), q (12\%), s (2\%) Mex, S, cv, f, qtf

\noindent\textbf{Elements of the pgp:} Preposition (p), Prepositional Temperer (pt), Completive~(c)

\subsection{Quality Group}
\textbf{Possible Relative Order of Elements in the Unit Structure:}\\ \noindent 
\&~|qld~|qlq~|et~|dt~|at~|a~|dt~|s~|f~|s~|e

\noindent\textbf{Filling probabilities of the qgp:} c (38\%), m (36\%), A (24\%), sd (0.5\%), Mex, Xex, od, q, dt, at, p, S

\noindent\textbf{Elements of the qlgp:} Quality Group Deictic (qld), Quality Group Quantifier (qlq), Emphasizing Temperer (et), Degree Temperer (dt), Adjunctival Temperer (at), Apex (a), Scope (s), Finisher (f)

\subsection{Quantity Group}
\textbf{Possible Relative Order of Elements in the Unit Structure:} \\ \noindent
ad~|am~|qtf~|e
\noindent\textbf{Filling probabilities of the qtgp:} qd (85\%), A (8\%), dt (6\%), B, p, ad, fd, sd
\noindent\textbf{Elements of the qtgp} Adjustor (ad), Amount (am), Quantity Finisher (qf)
\subsection{Genitive Cluster}
\textbf{Possible Relative Order of Elements in the Unit Structure:} \\ \noindent
\&~|po~|g~|o~|e

\noindent\textbf{Filling probabilities of the gencl:} dd (99\%), h, m, qld

\noindent\textbf{Elements of the gencl:} Possessor (po), Genitive Element (g), Own Element (o)

\section{Sydney Syntax}
\subsection{Logical}

\noindent\textbf{Possible Relative Order of Elements in the Unit Structure:} \\ \noindent
Pre-Modifier~|Head~|Post-Modifier

\subsection{Textual}
\noindent\textbf{Possible Relative Order of Elements in the Clause Structure:} \\ \noindent
Theme~|Rheme \\
New~|Given~|New

\subsection{Interactional}
\noindent\textbf{Possible Relative Order of Elements in the Clause Structure:} \\ \noindent
Residue~|Mood~|Residue~|Mood tag \\
Adjunct~|Complement~|Finite~|Subject~|Finite~|Adjunct~|Predicator~|Complement|~Adjunct

\subsection{Experiential}
\noindent\textbf{Possible Relative Order of Elements in the Clause Structure:} \\ \noindent
Circumstance~|Participant~|Circumstance~|Process|~Participant~|Circumstance

\noindent\textbf{Possible Relative Order of Elements in the Nominal Group Structure:} \\ \noindent
Deictic~|Numerative~|Epithet~|~Classifier|~Thing~|Qualifier

\noindent\textbf{Possible Relative Order of Elements in the Verbal Group Structure:} \\ \noindent
Finite~|Marker~|Auxiliary~|Event

\noindent\textbf{Possible Relative Order of Elements in the Adverbial and Preposition Group Structure:} \noindent Modifier~|Head~|Post-Modifier

\noindent\textbf{Possible Relative Order of Elements in the Prepositional Phrase Structure:} \\ \noindent
Predicator~|Complement \\ \noindent
Process~|Range \\

\subsection{Taxis}
\noindent\textbf{Possible Relative Order of Elements in the Parataxis Structure:} \\ \noindent
Initiating~|Continuing \\
\noindent\textbf{Possible Relative Order of Elements in the Hypoataxis Structure:} \\ \noindent
Dependent~|Dominant~|Dependent\\

\chapter{Stanford Dependency schema}
\label{ch:stanfordDepRel}
The Stanford dependency relations as defined in Stanford typed dependencies manual \citep{Marneffe2008}

\begin{figure}[!ht]
    \centering
    \begin{forest}
        for tree={
            font=\ttfamily,
            grow'=0,
            child anchor=west,
            parent anchor=south,
            anchor=west,
            calign=first,
            edge path={
                \noexpand\path [draw, \forestoption{edge}]
                (!u.south west) +(7.5pt,0) |- node[fill,inner sep=1.25pt] {} (.child anchor)\forestoption{edge label};
            },
            before typesetting nodes={
                if n=1
                {insert before={[,phantom]}}
                {}
            },
            fit=band,
            before computing xy={l=15pt},
        }
        [dep - dependent
        [arg - argument 
        [agent - agent ]
        [comp - complement 
        [acomp - adjectival complement ]
        [attr - attributive ]
        [ccomp - clausal complement with internal subject ]
        [xcomp - clausal complement with external subject ]
        [compl - complementizer ]
        [obj - object 
        [dobj - direct object ]
        [iobj - indirect object ]
        [pobj - object of preposition ]
        ]
        [mark - marker (word introducing an advcl) ]
        [rel - relative (word introducing a rcmod) ]
        ]
        [subj - subject 
        [nsubj - nominal subject 
        [nsubjpass - passive nominal subject ]
        ]
        [csubj - clausal subject 
        [csubjpass - passive clausal subject ]
        ]
        ]
        ]
        [\dots]
        ] 
    \end{forest}
    \caption{The Stanford dependency scheme - part one}
    \label{fig:grStanford1}
\end{figure}

\begin{figure}[!ht]
    \centering
    \begin{forest}
        for tree={
            font=\ttfamily,
            grow'=0,
            child anchor=west,
            parent anchor=south,
            anchor=west,
            calign=first,
            edge path={
                \noexpand\path [draw, \forestoption{edge}]
                (!u.south west) +(7.5pt,0) |- node[fill,inner sep=1.25pt] {} (.child anchor)\forestoption{edge label};
            },
            before typesetting nodes={
                if n=1
                {insert before={[,phantom]}}
                {}
            },
            fit=band,
            before computing xy={l=15pt},
        }
        [dep - dependent
        [\dots]
        [mod - modifier 
        [abbrev - abbreviation modifier ]
        [amod - adjectival modifier ]
        [appos - appositional modifier ]
        [advcl - adverbial clause modifier ]
        [purpcl - purpose clause modifier ]
        [det - determiner ]
        [predet - predeterminer ]
        [preconj - preconjunct ]
        [infmod - infinitival modifier ]
        [partmod - participial modifier ]
        [advmod - adverbial modifier 
        [neg - negation modifier ]			
        ]
        [rcmod - relative clause modifier ]
        [quantmod - quantifier modifier ]
        [tmod - temporal modifier ]
        [measure - measure-phrase modifier ]
        [nn - noun compound modifier ]
        [num - numeric modifier ]
        [number - element of compound number ]
        [prep - prepositional modifier ]
        [poss - possession modifier ]
        [possessive - possessive modifier ('s) ]
        [prt - phrasal verb particle ]
        ]
        [\dots]
        ] 
    \end{forest}
    \caption{The Stanford dependency scheme - part two}
    \label{fig:grStanford2}
\end{figure}

\begin{figure}[!ht]
    \centering
    \begin{forest}
        for tree={
            font=\ttfamily,
            grow'=0,
            child anchor=west,
            parent anchor=south,
            anchor=west,
            calign=first,
            edge path={
                \noexpand\path [draw, \forestoption{edge}]
                (!u.south west) +(7.5pt,0) |- node[fill,inner sep=1.25pt] {} (.child anchor)\forestoption{edge label};
            },
            before typesetting nodes={
                if n=1
                {insert before={[,phantom]}}
                {}
            },
            fit=band,
            before computing xy={l=15pt},
        }
        [dep - dependent
        [\dots]
        [aux - auxiliary 
        [auxpass - passive auxiliary ]
        [cop - copula ]
        ]
        [cc - coordination ]
        [conj - conjunct ]
        [expl - expletive (expletive "there") ]
        [parataxis - parataxis ]
        [punct - punctuation ]
        [ref - referent ]
        [sdep - semantic dependent 
        [xsubj - controlling subject ]
        ]
        ] 
    \end{forest}
    \caption{The Stanford dependency scheme - part three}
    \label{fig:grStanford3}
\end{figure}

\chapter{Penn treebank tag-set}
\label{ch:PennTagset}

% Please add the following required packages to your document preamble:
% \usepackage{graphicx}
\begin{table}[!ht]
    \centering
    \resizebox{\textwidth}{!}{%
        \begin{tabular}{|l|l|l|}
            \hline
            \textbf{Tag} & \textbf{Description}                      & \textbf{Example}           \\ \hline
            CC           & conjunction, coordinating                 & and, or, but               \\ \hline
            CD           & cardinal number                           & five, three, 13\%          \\ \hline
            DT           & determiner                                & the, a, these              \\ \hline
            EX           & existential there                         & there were six boys        \\ \hline
            FW           & foreign word                              & mais                       \\ \hline
            IN           & conjunction, subordinating or preposition & of, on, before, unless     \\ \hline
            JJ           & adjective                                 & nice, easy                 \\ \hline
            JJR          & adjective, comparative                    & nicer, easier              \\ \hline
            JJS          & adjective, superlative                    & nicest, easiest            \\ \hline
            LS           & list item marker                          &                            \\ \hline
            MD           & verb, modal auxillary                     & may, should                \\ \hline
            NN           & noun, singular or mass                    & tiger, chair, laughter     \\ \hline
            NNS          & noun, plural                              & tigers, chairs, insects    \\ \hline
            NNP          & noun, proper singular                     & Germany, God, Alice        \\ \hline
            NNPS         & noun, proper plural                       & we met two Christmases ago \\ \hline
            PDT          & predeterminer                             & both his children          \\ \hline
            POS          & possessive ending                         & 's                         \\ \hline
            PRP          & pronoun, personal                         & me, you, it                \\ \hline
            PRP\$        & pronoun, possessive                       & my, your, our              \\ \hline
            RB           & adverb                                    & extremely, loudly, hard    \\ \hline
            RBR          & adverb, comparative                       & better                     \\ \hline
            RBS          & adverb, superlative                       & best                       \\ \hline
            RP           & adverb, particle                          & about, off, up             \\ \hline
            SYM          & symbol                                    & \%                         \\ \hline
            TO           & infinitival to                            & what to do?                \\ \hline
            UH           & interjection                              & oh, oops, gosh             \\ \hline
            VB           & verb, base form                           & think                      \\ \hline
            VBZ          & verb, 3rd person singular present         & she thinks                 \\ \hline
            VBP          & verb, non-3rd person singular present     & I think                    \\ \hline
            VBD          & verb, past tense                          & they thought               \\ \hline
            VBN          & verb, past participle                     & a sunken ship              \\ \hline
            VBG          & verb, gerund or present participle        & thinking is fun            \\ \hline
            WDT          & wh-determiner                             & which, whatever, whichever \\ \hline
            WP           & wh-pronoun, personal                      & what, who, whom            \\ \hline
            WP\$         & wh-pronoun, possessive                    & whose, whosever            \\ \hline
            WRB          & wh-adverb                                 & where, when                \\ \hline
            .            & punctuation mark, sentence closer         & .;?*                       \\ \hline
            ,            & punctuation mark, comma                   & ,                          \\ \hline
            :            & punctuation mark, colon                   & :                          \\ \hline
            (            & contextual separator, left paren          & (                          \\ \hline
            )            & contextual separator, right paren         & )                          \\ \hline
        \end{tabular}%
    }
    \caption{Penn Treebank tag set}
    \label{tab:penntagset}
\end{table}