\chapter{SFL Syntactic Overview}
\label{ch:syntax-overview}

\section{Cardiff Syntax}
\noindent\textbf{Elements found in all groups:} Linker (\&), Inferer (I), Starter (st), Ender (e)

\noindent\textbf{Units:} Sentence ($\Sigma$), Clause (Cl), Nominal Group (ngp), Prepositional Group (pgp), Quality Group (qlgp), Quantity Group (qtgp), Genitive Cluster (gencl)

\subsection{Clause}

\textbf{Relative Order of Elements in the Unit Structure:}\\ \noindent
\&~|B~|L~|F~|A~|C~|O~|S~|O~|N~|A~|I~|X~|M~|Mex~|C~|A~|V~|E

\noindent\textbf{Clause May fill:}
$\Sigma$ (85\%), C (7\%), A (4\%), Q (2\%), f (0.5\%), s, qtf, S, m, cv, po

\noindent\textbf{Elements of the Clause:}
Adjunct (A), Binder (B), Complement (C), Formulaic Element (F), Infinitive Element (I), Let Element (L),Main Verb (M), Main Verb Extension (Mex), Negator (N), Operator (O), Subject (S), Vocative (V), Auxiliary Verb (A), X extension (Xex), Linker (\&), Starter (St), Ender(E)

\subsection{Nominal Group}
\textbf{Possible Relative Order of Elements in the Unit Structure:} \\ \noindent
\&~|rd~|v~|pd~|v~|qd~|v~|sd~|v~|od~|v~|td~|v~|dd~|m~|h~|q~|e

\noindent\textbf{Filling probabilities of the ngp:} S (45\%), C (32\%), cv (15\%), A (3\%), m (2\%), Mex, V, rd, pd, fd, qd, td, q, dt, po

\noindent\textbf{Elements of the ngp:} Representational determiner (rd), Selector (v), Partitive Determiner (pd), Fractionative Determiner (fd), Quantifying Determiner (qd), Superlative Determiner (sd), Ordinative Determiner (od), Qualifier-Introducing Determiner (qid), Typic Determiner (td), Deictic Determiner (dd), Modifier (m), Head (h), Qualifier (q)

\subsection{Prepositional Group}
\textbf{Possible Relative Order of Elements in the Unit Structure:} \\ \noindent
\&~|pt~|p~|cv~|p~|e

\noindent\textbf{Filling Probabilities of the pgp:} C (55\%), a (30\%), q (12\%), s (2\%) Mex, S, cv, f, qtf

\noindent\textbf{Elements of the pgp:} Preposition (p), Prepositional Temperer (pt), Completive~(c)

\subsection{Quality Group}
\textbf{Possible Relative Order of Elements in the Unit Structure:}\\ \noindent 
\&~|qld~|qlq~|et~|dt~|at~|a~|dt~|s~|f~|s~|e

\noindent\textbf{Filling probabilities of the qgp:} c (38\%), m (36\%), A (24\%), sd (0.5\%), Mex, Xex, od, q, dt, at, p, S

\noindent\textbf{Elements of the qlgp:} Quality Group Deictic (qld), Quality Group Quantifier (qlq), Emphasizing Temperer (et), Degree Temperer (dt), Adjunctival Temperer (at), Apex (a), Scope (s), Finisher (f)

\subsection{Quantity Group}
\textbf{Possible Relative Order of Elements in the Unit Structure:} \\ \noindent
ad~|am~|qtf~|e
\noindent\textbf{Filling probabilities of the qtgp:} qd (85\%), A (8\%), dt (6\%), B, p, ad, fd, sd
\noindent\textbf{Elements of the qtgp} Adjustor (ad), Amount (am), Quantity Finisher (qf)
\subsection{Genitive Cluster}
\textbf{Possible Relative Order of Elements in the Unit Structure:} \\ \noindent
\&~|po~|g~|o~|e

\noindent\textbf{Filling probabilities of the gencl:} dd (99\%), h, m, qld

\noindent\textbf{Elements of the gencl:} Possessor (po), Genitive Element (g), Own Element (o)

\section{Sydney Syntax}
\subsection{Logical}

\noindent\textbf{Possible Relative Order of Elements in the Unit Structure:} \\ \noindent
Pre-Modifier~|Head~|Post-Modifier

\subsection{Textual}
\noindent\textbf{Possible Relative Order of Elements in the Clause Structure:} \\ \noindent
Theme~|Rheme \\
New~|Given~|New

\subsection{Interactional}
\noindent\textbf{Possible Relative Order of Elements in the Clause Structure:} \\ \noindent
Residue~|Mood~|Residue~|Mood tag \\
Adjunct~|Complement~|Finite~|Subject~|Finite~|Adjunct~|Predicator~|Complement|~Adjunct

\subsection{Experiential}
\noindent\textbf{Possible Relative Order of Elements in the Clause Structure:} \\ \noindent
Circumstance~|Participant~|Circumstance~|Process|~Participant~|Circumstance

\noindent\textbf{Possible Relative Order of Elements in the Nominal Group Structure:} \\ \noindent
Deictic~|Numerative~|Epithet~|~Classifier|~Thing~|Qualifier

\noindent\textbf{Possible Relative Order of Elements in the Verbal Group Structure:} \\ \noindent
Finite~|Marker~|Auxiliary~|Event

\noindent\textbf{Possible Relative Order of Elements in the Adverbial and Preposition Group Structure:} \noindent Modifier~|Head~|Post-Modifier

\noindent\textbf{Possible Relative Order of Elements in the Prepositional Phrase Structure:} \\ \noindent
Predicator~|Complement \\ \noindent
Process~|Range \\

\subsection{Taxis}
\noindent\textbf{Possible Relative Order of Elements in the Parataxis Structure:} \\ \noindent
Initiating~|Continuing \\
\noindent\textbf{Possible Relative Order of Elements in the Hypoataxis Structure:} \\ \noindent
Dependent~|Dominant~|Dependent\\
