\chapter{Enrichment}

\begin{figure}[!ht]%{0.4\textwidth}
		\centering
		\begin{tikzpicture}[tree-style,,level distance=4em,]
		\node (cla) [pattern-node] {class:clause}
		child { node (sub)[pattern-node] {element:Subject,\\ precede:f1}}		
		child { node (f)[pattern-node] {element:Finite,\\id:f1}};
		\end{tikzpicture}
		\caption{Declarative mood pattern graph with relative element order}
		\label{fig:pg-declarative2}
\end{figure}

\begin{figure}[!ht]
	\centering
	\begin{tikzpicture}[tree-style,,level distance=4em,]
	\node (cla) [pattern-node] {class:clause}
	child { node (f)[pattern-node] {element:Finite,\\precede:s1}}
	child { node (sub)[pattern-node] {element:Subject,\\id:s1}}	;	
%	child { node (mv)[pattern-node] {element:main verb,\\id:mv1}};
	\end{tikzpicture}
	\caption{Pattern graph for Yes/No interrogative mood}
	\label{fig:pg-interrogative}
\end{figure}

\begin{figure}[hbtp]
    \centering
    \scalebox{0.8}{
    \begin{tikzpicture}[tree-style,level distance=5em, level 1/.style={sibling distance=12.5em},]
    \node[pattern-node, anchor=center](vb1){element:clause}
    	child {node[pattern-node,xshift=0em] {element:subject,\\operation:update,\\ arg1:\{participant:agent\}} edge from parent node[above] {}}
    	child {node[pattern-node,xshift=0em] {element:complement,\\operation:update,\\ arg1:\{participant:posessed\}} edge from parent node[below right] {}}
    	child {node[pattern-node-negative, xshift=0em] {element:complement} edge from parent node[above] {}};
    \end{tikzpicture}
    }
    \caption{PG for inserting agent and possessed participant roles to subject and complement nodes only if there is no second complement.}
    \label{fig:gp4}
    \end{figure}
    
    
\section{Enrichment}    

    \begin{figure}[!ht]
        \centering
        \scalebox{0.7}{
        \begin{tikzpicture}[
        tree-style, 
    %    edge-style,
        level 1/.style={sibling distance=10em},
        level 2/.style={sibling distance=8em},
        level distance = 4em
        ]
        \node[pattern-node]{class: clause}
        child {node[pattern-node]{element: Subject\\word: He} }
        child {node[pattern-node]{element: Main verb\\word: gave} }
        child {node[pattern-node]{element: Complement\\class: nominal group}  {}
            child {node[pattern-node]{element: Deictic\\word: the} }
            child {node[pattern-node]{element: Thing\\word: cake} }
        }
        child {node[pattern-node]{element: Adjunct\\word: away}  }
        ;
        \end{tikzpicture}
    }
        \caption{Constituency graph corresponding to Example \ref{ex:transitivity1}}
        \label{fig:cg-transitive1}
    \end{figure}
    
    
    \begin{figure}[!ht]
    \centering
    \scalebox{0.7}{
    \begin{tikzpicture}[tree-style, level 1/.style={sibling distance=14em}, level distance = 5em]
    \node[pattern-node]{class:clause}
    	child {node[pattern-node, xshift=1.5em]{element: Subject\\operation: update\\arg1:\{participant: agent\}} }
       	child {node[pattern-node]{element: Main verb\\lemma: give\\operation: update\\arg1:\{process: possessive\}} }
        child {node[pattern-node, xshift=1em]{element: Complement\\operation: update\\arg1:\{participant: affected-possessed\}}};
    \end{tikzpicture}
    }
    \caption{A graph pattern for inserting agent and affected-possessed participant roles}
    \label{fig:pg-transitive1}
    \end{figure}
    \begin{figure}[!ht]
        \centering
        \scalebox{0.7}{
            \begin{tikzpicture}[
            tree-style, 
            %    edge-style,
            level 1/.style={sibling distance=12em},
            level 2/.style={sibling distance=8em},
            level distance = 5em
            ]
            \node[pattern-node]{class: clause}
            child {node[pattern-node, xshift=2em]{element: Subject\\\textit{participant: agent}\\word: He} }
            child {node[pattern-node]{element: Main verb\\\textit{process: possessive}\\word: gave} }
            child {node[pattern-node]{element: Complement\\\textit{participant: affected-possessed}\\class: nominal group}  {}
                child {node[pattern-node]{element: Deictic\\word: the} }
                child {node[pattern-node]{element: Thing\\word: cake} }
            }
            child {node[pattern-node]{element: Adjunct\\word: away}  }
            ;
            \end{tikzpicture}
        }
        \caption{The resulting constituency graph enriched with participant roles}
        \label{fig:cgg-transitive1}
    \end{figure}
    
    
\section{sec}

    \begin{figure}[!ht]
        \centering
        \scalebox{0.7}{
        \begin{tikzpicture}[tree-style,level distance=5em,] 
        \node[pattern-node, anchor=center] (c) {class:clause}
        child {node[pattern-node] (subj1) {element:Main Verb}};
        \end{tikzpicture}
        }
        \caption{A graph pattern for \textit{major} feature selection in Figure \ref{fig:mood-fragment}}
        \label{fig:mood-fragment-major}
    \end{figure}
    
    
    \begin{figure}[!ht]
        \centering
        \begin{dependency}[dep-style-narrow]
            \begin{deptext}[]
                PRP \& VBD \& DT \& NN \& RB\\ 
                He \& gave \& the \& cake \& away\\ 
            \end{deptext}
            \depedge[]{2}{1}{nsubj}
            \depedge{2}{4}{dobj}
            \depedge{2}{5}{advmod}
            \depedge{4}{3}{det}
        \end{dependency}
        \caption{Dependency representation } %Example of parent-child hierarchical dependency
        \label{fig:dependency-dg-ex}
    \end{figure}